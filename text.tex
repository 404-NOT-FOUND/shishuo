
\chapter{德行第一}
\label{cha:de_xing_di_yi}

\begin{paracol}{2}

%% ----------------------------------------------------------------------------
\switchcolumn[0]*[\section{}]

陈仲举言为士则,行为世范%
\footnote{%
    陈仲举:
    名蕃,字仲举,
    东汉时期名臣,与窦武、刘淑合称「三君」。
    当时宦官专权,他与大将军窦武谋诛宦官。
    未成,反被害。
}%
。
登车揽辔,有澄清天下之志%
\footnote{辔(pèi):牲口的缰绳。}%
。
为豫章太守%
\footnote{%
    豫章:豫章郡,首府在南昌(今江西省南昌县)。
    太守:郡的行政长官。
}%
,
至,便问徐孺子所在,欲先看\mbox{之%
\footnote{徐孺子:名稚,字孺子,东汉豫章南昌人,是当时的名士、隐士。}}%
。
主薄白:「群情欲府君先入廨%
\footnote{%
    府君:对太守的称呼。太守办公的地方称府,所以称太守为府君。
    廨(xiè):官署;衙门。
}%
。」
陈曰:「
    武王式商容之闾%
    \footnote{%
        式:示范,表彰。
        商容:商纣时的大夫,当时被认为是贤人。
        闾:指里巷。
    }%
    ,席不暇暖。
    吾之礼贤,有何不可!
」

\switchcolumn

陈仲举的言谈举止都可谓当世人的楷模。
他走马上任,巡察各地,有志肃清吏治,使九州海晏河清。

出任豫章郡太守时,
他刚一到就要打听徐孺子的住处,想先去拜访他。
主簿对他说:「
    大家恳请府君您先到衙门上去。
」
陈仲举说:「
    周武王刚打下殷的江山,
    连休息也顾不上。
    就要专门去商容家拜访。
    我以尊爱贤人为重,不先去衙门,又有什么不应该!
」


%% ----------------------------------------------------------------------------
\switchcolumn*[\section{}]

周子居常云%
\footnote{%
    周子居:
    名乘,字子居,东汉时人,不畏强暴。
    陈仲举曾赞他为「治国之器」。
}%
:「
    吾时月不见黄叔度%
    \footnote{%
        时月:时日。
        黄叔度:名宪,字叔度,出身贫寒,有德行;得到时人赞誉。
    }%
    ,则鄙吝之心已复生矣。
」

\switchcolumn

周子居常常说:「
    我只要有一段时间没有见到黄叔度,
    庸俗吝啬的小人心思就又蠢蠢欲动了起来。
」


%% ----------------------------------------------------------------------------
\switchcolumn*[\section{}]

郭林宗至汝南,造袁奉高%
\footnote{%
    郭林宗:名泰,字林宗,东汉人,博学有德,为时人所重。
    袁奉高:名阆(làng),字奉高,和黄叔度同为汝南郡慎阳人;
            多次辞谢官府任命,也很有名望。
            曾为汝南郡功曹,后为太尉属官。
}%
,车不停轨,鸾不辍轭%
\footnote{%
    轨:车子两轮之间的距离,其宽度为古制八尺,后引申为车辙。
    辍(chuò):停止。
    鸾:装饰在车上的铃子,这里指车子。
    轭(è):驾车时搁在牛马颈上的曲木。
}%
;
诣黄叔度,乃弥日信宿%
\footnote{%
    弥日:终日;整天。
    信宿:连宿两夜。
}%。
人问其故,
林宗曰:「
    叔度汪汪如万顷之陂,澄之不清,扰之不浊%
    \footnote{%
        陂(bēi):湖泊。
    }%
    ,其器深广,难测量也。
」

\switchcolumn

郭林宗到汝南郡
拜访袁奉高时,刚见面一会儿就走了,好像根本没停下车来似的;
待到他去拜访黄叔度,却一住就是两天。
别人问他缘故,
他说:「
    黄叔度就好比万顷的湖泊一般,澄不清也搅不浑;
    他深广的气量,着实难以测量!
」


%% ----------------------------------------------------------------------------
\switchcolumn*[\section{}]

李元礼风格秀整,高自标持,欲以天下名教是非为己任%
\footnote{%
    李元礼:
            名膺(yīng),字元礼,东汉人,曾任司隶校尉。
            当时朝廷纲纪废弛,他却独持法度,以声名自高。
            后谋诛宦官未成,被杀。
    风格秀整:风度出众。品性端庄。
    高自标持:自视甚高;很自负。
    名教:以儒家所主张的正名定分为准则的礼教。
}%。
后进之士,有升其堂者%
\footnote{%
    升其堂:登上他的厅堂,指有机会接受教诲。
}%
,皆以为登龙门%
\footnote{%
    龙门:在山西省河津县西北。
          那里水位落差很大,传说龟鱼不能逆水而上,
          有能游上去的,就会变成龙。
}%
。

\switchcolumn

李元礼风度出众,品性端庄,自视甚高。
他把在全国推行儒家礼教、辨明是非看成自己的责任。
后辈的读书人,如果有能进入他家聆听教诲的,都认为自己像是登上了龙门。


%% ----------------------------------------------------------------------------
\switchcolumn*[\section{}\label{sec:李元礼赞贤}]

李元礼尝叹荀淑、钟皓曰%
\footnote{%
    荀淑:字季和,东汉颖川郡人,曾任朗陵侯相。
          他和钟皓(字季明)两人都以清高有德,名重当时。
}%
:「
    荀君清识难尚,钟君至德可师%
    \footnote{%
        尚:超过。
    }%
。」

\switchcolumn

李元礼赞叹荀淑和钟皓二人,说:\\「
    荀君的见识之高明,是世人所望尘莫及的;
    而钟君的德行之高尚却是人们可以仿效的。
」


%% ----------------------------------------------------------------------------
\switchcolumn*[\section{}]

陈太丘诣荀朗陵%
\footnote{%
    陈太丘:名寔,字仲弓,曾任太丘县长,所以称陈太丘。古代常以官名称人。
    荀朗陵:指荀淑(见\ref{sec:李元礼赞贤})。
}%
,贫俭无仆役,
乃使元方将车,季方持杖后从%
\footnote{%
    元方、季方:都是陈寔的儿子。
                元方是长子,名纪,字元方;
                季方是少子,名湛,字季方。
                父子三人才德兼备,知名于时。
}%
。
长文尚小,载着车中%
\footnote{%
    长文:陈寔的孙子陈群。
}%。
既至,荀使叔慈应门,慈明行酒,余六龙下食,文若亦小,坐着膝前%
\footnote{%
    叔慈、慈明、六龙:苟淑有八个儿子,号称八龙。
                      叔慈、慈明是他两个儿子的名字,
                      其余六人就是这里所说的六龙了。
    文若:荀淑的孙子荀或。
    下食:上莱。
    膝前:膝上。
}%。
于时太史奏 :「真人东行%
\footnote{%
    太史:官名,主要掌管天文历法。
    真人:修真得道的人。此应指某星。
}%
。」

\switchcolumn

太丘县长陈寔去拜访朗陵侯荀淑。
因为家中清苦节俭,没有仆役侍候,
陈寔就让长子元方驾车送他,少子季方拿着手杖跟在车后。
孙子长文年纪还小,就一起坐在车上。
到了荀家,
荀淑让叔慈迎接客人,让慈明敬酒,其余六个儿子管上菜。
孙子文若也还小,就坐在荀淑膝上。
与此同时,太史启奏朝廷说:「星相发生变化,真人星向东移去了%
\footnote{%
    古时人们认为星相变化和世上的重大事件有关。
    作者此处意在说明陈荀两家贤人聚首的事件引发了真人星的星相变化。
}%
。」


%% ----------------------------------------------------------------------------
\switchcolumn*[\section{}]

客有问陈季方:「足下家君太丘有何功德而荷天下重名%
\footnote{%
    家君:父亲。对自己或他人父亲的尊称。
    荷(hè):担当;承受。
}%
?」
季方曰:「
    吾家君譬如桂树生泰山之阿,
    上有万侧之高,下有不测之深%
    \footnote{%
        阿(ē):山的拐角儿。
        侧(rèn):长度单位,一侧等于七尺或八尺
    }%
    ;
    上为甘露所沾,下为渊泉所润%
    \footnote{%
        渊泉:深泉。
    }%
    。
    当斯之时,桂树焉知泰山之高,渊泉之深!
    不知有功德与无也!
」

\switchcolumn

有人问陈季方:「
    令尊区区的县长,敢问他何德何能,来担负起享誉天下的声望?
」
季方说:「
    家父就好比生长在泰山一角的桂树;
    仰头是万丈的高峰,低眉则有无底的深渊;
    他上要凭靠雨露的浇灌,下要蒙受深泉的滋养,
    才得以茁壮。
    此情此景之下,
    桂树怎么会知道泰山有多高,渊泉又有多深呢!
    如此说来,我又哪知家父功业几何?
」


%% ----------------------------------------------------------------------------
\switchcolumn*[\section{}]

陈元方子长文,有英才,
与季方子孝先
各论其父功德,争之不能决,咨于太丘。
太丘曰:「元方难为兄,季方难为弟%
\footnote{%
    按:这两句不会是陈寔的原话,因为父亲不会称呼儿子的字。
}%
。」

\switchcolumn

陈元方的儿子长文才华出众。
他与叔叔季方的儿子孝先各自论述他们父亲的功德,
争执不下,于是请祖父陈太丘裁决。
太丘说:「
    % 季方很好,好得元方做哥哥都难做;
    % 元方也很好,好得季方弟弟都难做。
    你们的父亲都很优秀,
    在这方面,可以说是不分伯仲了。
」


%% ----------------------------------------------------------------------------
\switchcolumn*[\section{}]

荀巨伯远看友人疾%
\footnote{%
    荀巨伯:东汉人。
}%
,
值胡贼攻郡%
\footnote{%
    胡:古时西方、北方各少数民族统称胡。
}%
,
友人语巨伯曰:「
    吾今死矣,子可去%
    \footnote{%
        子:对对方的尊称,相当于「您」。
    }%
    !
」
巨伯曰:「
    远来相视,子令吾去;
    败义以求生,岂荀巨伯所行邪%
    \footnote{%
        邪:文言疑问词,相当于「呢」或「吗」。
    }%
    !
」
贼既至,谓巨伯曰:「
    大军至,一郡尽空,汝何男子,而敢独止?
」
巨伯曰:「
    友人有疾,不忍委之%
    \footnote{%
        委:舍弃,抛弃。
    }%
    ,宁以吾身代友人命。
」
贼相谓曰:「
    吾辈无义之人,而入有义之国。
」
遂班军而还,一郡并获全%
\footnote{%
    班军:班师;调回出征的军队。
}%
。

\switchcolumn

荀巨伯远道来探望罹病的友人,
正碰见胡人入犯,攻打城郡。
朋友对巨伯说:「
    看来我是大期将至了。您快先逃走吧!
」
巨伯说道:「
    % 我远道而来,就为了看看您的病情,
    我远道来看你,
    您却叫我独自逃生;
    我荀巨伯能是这样不仁不义、苟且偷生的人吗!
」
胡军打到郡内,喝问巨伯:「
    我们大军一路杀进来,
    郡里的人早逃得一干二净。
    你是何方神圣,
    居然敢留下来送死?
」
巨伯道:「
    我朋友身患疾病,走不了;
    我不忍心弃他而去,
    愿一命换一命,
    保他一条生路。
」
胡人面面相觑,说道:「
    看来我们是一帮无情无义之人
    闯入了这有情有义的国度呀。
」
随即回调大军,返回了家乡。
这个城郡也因此得以保全。


%% ----------------------------------------------------------------------------
\switchcolumn*[\section{}]

华歆遇子弟
甚整,虽闲室之内,严若朝典%
\footnote{%
    华歆:字子鱼,汉朝人。
          同邴原、管宁一起在外求学,
          三人很友好。
          当时人们称他们三人为一龙,
          说华歆是龙头,
          管宁是龙腹,
          邴原是龙尾。
    闲室:静室,这里指家庭。
    朝典:朝廷的礼仪。
}%
。
陈元方兄弟恣柔爱之道%
\footnote{%
    柔爱:和睦、友爱。
}%
,而二门之里,两不失雍熙之轨焉%
\footnote{%
    雍熙:和乐。
    轨:法则、准则。
}%
。

\switchcolumn

华歆对待晚辈非常地板正,
即便是在家里,也严格地遵循朝堂的礼仪。
陈元方兄弟相亲相爱,
但也把各自的家里照顾得很好,
家中和乐融融。


%% ----------------------------------------------------------------------------
\switchcolumn*[\section{}]

管宁、华歆共园中锄菜,
见地有片金,
管挥锄与瓦石不异,华捉而掷去之。
又尝同席读书,
有乘轩冕过门者%
\footnote{%
    轩冕:大夫以上的贵族坐的车和戴的礼帽。
}%,
宁读如故,歆废书出看,
宁割席分坐,
曰:「子非吾友也 !」

\switchcolumn

\end{paracol}

