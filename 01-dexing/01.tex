
\switchcolumn[0]*[\section{}]

陈仲举言为士则,行为世范%
\footnote{%
    陈仲举:
    名蕃,字仲举,
    东汉时期名臣,与窦武、刘淑合称「三君」。
    当时宦官专权,他与大将军窦武谋诛宦官。
    未成,反被害。
}%
。
登车揽辔,有澄清天下之志%
\footnote{辔(pèi):牲口的缰绳。}%
。
为豫章太守%
\footnote{%
    豫章:豫章郡,首府在南昌(今江西省南昌县)。
    太守:郡的行政长官。
}%
,
至,便问徐孺子所在,欲先看\mbox{之%
\footnote{徐孺子:名稚,字孺子,东汉豫章南昌人,是当时的名士、隐士。}}%
。
主薄白:「群情欲府君先入\mbox{廨%
\footnote{%
    府君:对太守的称呼。太守办公的地方称府,所以称太守为府君。
    廨(xiè):官署;衙门。
}}%
。」
陈曰:「
    武王式商容之闾%
    \footnote{%
        式:示范,表彰。
        商容:商纣时的大夫,当时被认为是贤人。
        闾:指里巷。
    }%
    ,席不暇暖。
    吾之礼贤,有何不可!
」

%% ----------------------------------------------------------------------------
\switchcolumn

陈仲举的言谈举止都可谓当世人的楷模。
他走马上任,巡察各地,有志肃清吏治,使九州海晏河清。

出任豫章郡太守时,
他刚一到就要打听徐孺子的住处,想先去拜访他。
主簿对他说:「
    大家恳请府君您先到衙门上去。
」
陈仲举说:「
    周武王刚打下殷的江山,
    连休息也顾不上。
    就要专门去商容家拜访。
    我以尊爱贤人为重,不先去衙门,又有什么不应该!
」

