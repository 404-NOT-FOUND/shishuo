
\switchcolumn*[\section{}]

顾荣在洛阳,
尝应人请%
\footnote{%
    顾荣:字彦先,东吴丞相顾雍之孙,
          西晋末年拥护司马氏政权南渡的江南士族首脑。
          弱冠之后就仕于孙吴,吴亡后,与陆机、陆云同入洛,号为「洛阳三俊」。
          后为琅玡王司马睿(晋元帝)安东军司,加散骑常侍,
          司马睿但凡有谋划,都与顾荣商议。
}%
,
觉行炙人有欲炙之色,因辍己施焉%
\footnote{%
    行炙人:传递菜肴的仆役。
    炙:烤肉。
}%
,
同坐嗤之%
\footnote{%
    嗤(chī):讥笑。
}%
。
荣曰:「
    岂有终日执之,而不知其味者乎?
」
后遭乱渡江,每经危急,常有一人左右己%
\footnote{%
    左右:帮助。
}%
,
问其所以,乃受炙人也。

%% ----------------------------------------------------------------------------
\switchcolumn

% %% Jy
% 有一次顾荣在洛阳的时候
% 受邀赴宴。
% 他发现上菜的下人面露馋色,
% 于是放下碗筷,把自己的那一份分给他。
% 在座的人都笑他。
% 顾荣说:「
%     他天天端着这些菜,
%     却还不知道菜肴的味道,
%     这怎么行呢?
% 」
% 后来发生永嘉之乱,顾荣渡江的时候
% 几经危难,
% 都为一个人所相助。
% 顾荣一问之下,才知道
% 这就是当年受他恩惠的那个下人。

% %% 妖
% 顾荣在洛阳的时候,
% 曾受邀去赴宴。
% 他发现传菜的下人对着烤肉露出垂涎欲滴的神色,就放下碗筷,把自己的那份分给了他。
% 在座的人对此都不以为然,面露嗤笑。
% 顾荣说:“
% 哪里有天天端着这些菜,却从来没有尝过它们味道的道理呢?”
% 后来,遭遇永嘉之乱时,顾荣要渡江,每每遇见危机,总能受人相助,从而化险为夷。
% 细问之下,顾荣才知这人就是当年那个传菜人。
