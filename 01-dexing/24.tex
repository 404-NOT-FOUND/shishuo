
\switchcolumn*[\section{}]

郗公值永嘉丧乱,在乡里,甚穷馁。
乡人以公名德,传共饴之。
公常携兄子迈及外生周翼二小儿往食,
乡人曰:「
    各自饥困,
    以君之贤,欲共济君耳,
    恐不能兼有所存。
」
公于是独往食,
辄含饭两颊边,还,吐与二儿。
后并得存,同过江。
郗公亡,
翼为剡县,解职归,席苫于公灵床头,心丧终三年。

%% ----------------------------------------------------------------------------
\switchcolumn

% %% Jy

% %% 妖
% 永嘉之乱期间,郗公被困乡里,穷困潦倒。
% 同乡人敬重他的名声与德行,就轮流供养他。
% 郗公经常带着他的侄子郗迈与外甥周翼两个小孩一同去吃饭。
% 乡里的人说:“
%  大家日子都不好过,
%  因为您素有贤名,所以大伙挤出粮食来供应您,
%  但恐怕不能再多养两个孩子了。”
% 于是郗公就独自去吃饭,
% 之后将饭含在两颊内,回家之后,吐出来给两个晚辈吃。
% 后来三个人都幸存了下来,一块到了江南。
% 郗公去世后,
% 周翼正担任剡县县令,于是辞官回去,在郗公灵前以子礼尽孝,寝苫枕块,守孝三年。
