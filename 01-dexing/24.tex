
\switchcolumn*[\section{}]

郗公值永嘉丧乱,在乡里,甚穷馁%
\footnote{%
    郗(xī)公:郗鉴,字道徽,以儒雅著名,
                过江后历任兖州刺史、司空、太尉。
    永嘉丧乱:晋怀帝永嘉年间(公元 307--312 年),
              正当八王之乱以后,政治腐败,民不聊生。
              至永嘉五年(公元 311 年),
              在山西称帝的匈奴贵族刘聪(国号汉)将领石勒、刘曜
              俘杀宰相王衍,攻破洛阳,俘怀帝,焚毁全城,史称永嘉之乱。
    穷:生活困难。
    馁(něi):饥饿。
}%
。
乡人以公名德,传共饴之%
\footnote{%
    传:轮流。
    饴(sì):通「饲」,给人吃。
}%
。
公常携兄子迈及外生周翼二小儿往食%
\footnote{%
    迈:字思远,有干世才略。
    周翼:字子卿,历任剡县令、青州刺史、少府卿,享年六十四岁。
    外生:外甥。
}%
。
乡人曰:「
    各自饥困,
    以君之贤,欲共济君耳,
    恐不能兼有所存。
」
公于是独往食,
辄含饭两颊边,还,吐与二儿。
后并得存,同过江%
\footnote{%
    过江:指渡过长江到江南。永嘉之乱时中原人士纷纷过江避难。
          后来镇守建康的琅臣子王司马睿即帝位,
          开始了东晋时代。
}%
。
郗公亡,
翼为剡县,解职归%
\footnote{%
    为剡(shàn)县:指做剡县县令。剡县,古属会稽郡(今浙江嵊县)。
}%
,席苫于公灵床头,心丧终三年%
\footnote{%
    席苫(shān):铺草垫子为席,坐、卧在上面。
                  古时父母死了,就要在草垫子上枕着土块睡,叫做「寝苫枕块。」
    灵床:为死者设置的坐卧用具。
    心丧:好象哀悼父母一样的做法而没有孝子之服。
          古时父母死,服丧三年;
          外亲死,服丧五个月。
          郗鉴是舅父,是外亲,周翼却守孝三年,所以称心丧。
}%
。

%% ----------------------------------------------------------------------------
\switchcolumn

% %% Jy
% 永嘉之乱的时候,郗公在乡里,生活非常贫苦,食不果腹。
% 乡里人尊敬他是有名有德的贤人,
% 便让他轮流到家里来吃饭。
% 郗公去的时候,总是带着他哥哥的孩子郗迈和外甥周翼
% 一同去吃。
% 乡里人对他说:「
%     大家家里也都挺困难的,
%     因为敬重您的贤德,
%     才决定一起帮忙周济您。
%     但是还要再带上别的人,
%     我们恐怕也承受不起了。
% 」
% 所以郗公只好独自去吃饭,
% 吃的时候将饭含在两边的脸颊里,
% 一等回家就吐出来喂给两个孩子。
% 后来他们都得以生存下来,
% 一同过了长江。
% 郗公去世时,
% 周翼正任剡县县令,
% 听说之后辞官回来,
% 在郗公灵床前寝苫枕块,
% 像哀悼自己的父母一般
% 为郗公守了足足三年的丧。

% %% 妖
% 永嘉之乱期间,郗公被困乡里,穷困潦倒。
% 同乡人敬重他的名声与德行,就轮流供养他。
% 郗公经常带着他的侄子郗迈与外甥周翼两个小孩一同去吃饭。
% 乡里的人说:「
%  大家日子都不好过,
%  因为您素有贤名,所以大伙挤出粮食来供应您,
%  但恐怕不能再多养两个孩子了。」
% 于是郗公就独自去吃饭,
% 之后将饭含在两颊内,回家之后,吐出来给两个晚辈吃。
% 后来三个人都幸存了下来,一块到了江南。
% 郗公去世后,
% 周翼正担任剡县县令,于是辞官回去,在郗公灵前以子礼尽孝,寝苫枕块,守孝三年。
