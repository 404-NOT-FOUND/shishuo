
\switchcolumn*[\section{}]

初,
桓南郡、扬广共说殷荆州,
宜夺殷觊南蛮
以自树%
\footnote{%
    桓南郡:桓玄,字敬道,
            继承了他父亲桓温的爵位,封为南郡公,
            和殷仲堪是好朋友。
    杨广:殷仲堪为荆州刺史时,
          任用杨广的弟弟杨佺(quán)期为司马。
          殷仲堪起兵反,
          把军旅之事全部交给佺期兄弟掌握。
    殷觊(jì):字伯道,
                任南蛮校尉,是掌管南蛮地区的长官。
                他是殷仲堪的堂兄,
                殷仲堪想邀他起兵谋反,殷觊不参加;
                杨广劝仲堪杀了殷觊,仲堪不同意。
                殷觊也自动让了位。
}%
。
觊亦即晓其旨。
尝因行散%
\footnote{%
    行散:魏晋士大夫喜欢服五石散,
          吃后要走路,以便散发,
          这叫行散。
}%
,
率尔去下舍,便不复还%
\footnote{%
    率尔:轻率;随便。
    下舍:住宅。
}%
,
内外无预知者。
意色萧然,
远同斗生之无愠%
\footnote{%
    萧然:悠闲的样子。
    斗生:指春秋时楚国令尹(宰相)斗穀於菟(dòugǔwūtú),字子文。
          据《论语·公冶长》说,
          他三次做令尹,没有一点高兴的神色;
          又三次被罢官,也没有一点怨恨的神色。
    愠(yùn):怨恨。
}%
。
时论以此多之。

%% ----------------------------------------------------------------------------
\switchcolumn

% %% Jy
% 殷仲堪响应王恭的号召,
% 起兵谋反的时候,
% 南郡公桓玄和杨广
% 一同劝殷仲堪,
% 让他夺下殷仲堪的弟弟殷觊的领地,
% 方便树立自己的权势。
% 殷觊很快便得知了哥哥的算盘。
% 有一次
% 趁着服用五石散后行散散步的时机,
% 悄悄地离开了家里,
% 再也没有回来过。
% 他身边也没有一个人事先知道这个打算。
% 他出走后,
% 神色是悠闲自在,
% 就好像当初
% 斗穀於菟罢官后,
% 毫不怨恨的神采一样。
% 这件事情使他受到当时人们的好评。

% %% 妖
% 起初,
% 桓南郡与扬广一同游说殷荆州,
% 劝他应该把堂弟殷觊所辖的南蛮之地夺下来,
% 以建立自己的势力范围。
% 殷觊随机也听说了这个消息。
% 就乘着服用五石散后散步的机会,
% 悄悄地离开了家里,之后一去不回。
% 他周围也没有任何人知道他打算这样做。
% 离开时,殷觊神色悠闲潇洒,
% 像春秋时楚国宰相斗生被三次罢相也无怨言那样。
% 当时的人们都因此称赞他。


