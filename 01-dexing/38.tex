
\switchcolumn*[\section{}]

范宣年八岁,
后园挑菜,误伤指,大啼%
\footnote{%
    范宜:字子宣,家境贫寒,崇尚儒家经典,以讲授儒学为业。
          居住在豫章郡,被召为太学博士、散骑郎,推辞不就。
    挑:挑挖;挖出来。
}%
。
人问:「痛邪?」
答曰:「
    非为痛,
    身体发肤,不敢毁伤%
    \footnote{%
        身体发肤,不敢毁伤:
            语出《孝经》「身体发肤,受之父母,不敢毁伤,孝之始也。」
            身,躯干;体,头和四肢。
    }%
    ,
    是以啼耳。
」
宣洁行廉约,
韩豫章遗绢百匹,不受%
\footnote{%
    韩豫章:韩伯,字康伯,历任豫章太守、丹杨尹、吏部尚书。
    遗(wèi):赠送。
}%
;
减五十匹,复不受。
如是减半,遂至一匹,既终不受。
韩后与范同载,
就车中裂二丈与范,云:「
    人宁可使妇无裈邪%
    \footnote{%
        裈(kūn):裤子。
    }%
?」
范笑而受之。

%% ----------------------------------------------------------------------------
\switchcolumn

% %% Jy
% 这年范子宣八岁,
% 有一天
% 他在后园里挖菜的时候,
% 一不小心
% 把手指划伤了,
% 范子宣于是就嚎啕大哭。
% 别人问他:「是不是很痛呀?」
% 子宣说:「
%     并不是因为多痛,
%     只是
%     身体发肤,受之父母,
%     我不小心弄伤了手指,
%     是不孝,
%     所以才哭了。
% 」
%
% 范子宣为人
% 十分清廉节俭。
% 豫章太守韩康伯
% 曾经送他百匹丝绢,
% 范子宣却不愿接受。
% 韩太守于是减下五十匹
% 再给他,
% 他还是不要。
% 就这么反复减半,
% 一直到只剩下一匹布送给范子宣,
% 他还是不接受。
% 后来有一次,
% 韩太守和范子宣同乘一辆车,
% 韩太守当着范子宣的面,
% 扯下二丈布来,
% 递给范子宣说:「
%     再怎么说,
%     你总不能让你家夫人
%     连条裤子都没得穿吧?
% 」
% 范子宣听后一笑,
% 这才接下了布。

% %% 妖

