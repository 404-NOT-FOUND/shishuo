
\switchcolumn*[\section{}]

客有问陈季方:「足下家君太丘有何功德而荷天下重名%
\footnote{%
    家君:父亲。对自己或他人父亲的尊称。
    荷(hè):担当;承受。
}%
?」
季方曰:\\「
    吾家君譬如桂树生泰山之阿,
    上有万侧之高,下有不测之深%
    \footnote{%
        阿(ē):山的拐角儿。
        侧(rèn):长度单位,一侧等于七尺或八尺
    }%
    ;
    上为甘露所沾,下为渊泉所润%
    \footnote{%
        渊泉:深泉。
    }%
    。
    当斯之时,桂树焉知泰山之高,渊泉之深!
    不知有功德与无也!
」

%% ----------------------------------------------------------------------------
\switchcolumn

有人问陈季方:「
    令尊区区的县长,敢问他何德何能,来担负起享誉天下的声望?
」
季方说:「
    家父就好比生长在泰山一角的桂树;
    仰头是万丈的高峰,低眉则有无底的深渊;
    他上要凭靠雨露的浇灌,下要蒙受深泉的滋养,
    才得以茁壮。
    此情此景之下,
    桂树怎么会知道泰山有多高,渊泉又有多深呢!
    如此说来,我又哪知家父功业几何?
」

