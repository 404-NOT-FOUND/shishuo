
\switchcolumn*[\section{}]

王仆射在江州%
\footnote{%
    王仆射(yè):王愉,字茂和,
                  曾出任江州刺史、都督江州及豫州之四郡军事。
                  这招致豫州刺史庾楷的怨恨,
                  庾楷就和桓玄、殷仲堪共推王恭为盟主,起兵反帝室。
                  这时王愉到任不久,没有准备,
                  就逃亡到临川,
                  被俘。
                  桓玄篡位后,
                  升他为尚书左仆射(尚书省的副职)。
}%
,
为殷、桓所逐,
奔窜豫章,存亡未测。
王绥在都%
\footnote{%
    王绥:字彦猷,
          王愉的儿子,
          在桓玄任太尉时,他任太尉右长史。
}%
,
既忧戚在貌,
居处饮食,每事有降。
时人谓为「
    试守孝子%
    \footnote{%
        试守孝子:等于说见习孝子。
                  官吏正式任命前,先主持其事以试其才能,
                  称为试守。
                  王绥在父亲存亡未测之时便做出居丧的样子,
                  所以人们模仿职官称谓,称他为试守孝子。
    }%
」。

%% ----------------------------------------------------------------------------
\switchcolumn

% %% Jy

% %% 妖

