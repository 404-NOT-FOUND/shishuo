
\switchcolumn*[\section{}]

王仆射在江州%
\footnote{%
    王仆射:王愉,字茂和,
            曾出任江州刺史、都督江州及豫州之四郡军事。
            这招致豫州刺史庾楷的怨恨,
            庾楷就和桓玄、殷仲堪共推王恭为盟主,起兵反帝室。
            这时王愉到任不久,没有准备,
            就逃亡到临川,
            被俘。
            桓玄篡位后,
            升他为尚书左仆射(尚书省的副职)。
}%
,
为殷、桓所逐,
奔窜豫章,存亡未测。
王绥在都%
\footnote{%
    王绥:字彦猷,
          王愉的儿子,
          在桓玄任太尉时,他任太尉右长史。
}%
,
既忧戚在貌,
居处饮食,每事有降。
时人谓为「
    试守孝子%
    \footnote{%
        试守孝子:等于说见习孝子。
                  官吏正式任命前,先主持其事以试其才能,
                  称为试守。
                  王绥在父亲存亡未测之时便做出居丧的样子,
                  所以人们模仿职官称谓,称他为试守孝子。
    }%
」。

%% ----------------------------------------------------------------------------
\switchcolumn

% %% Jy
% 尚书左仆射王茂和
% 在江州出任刺史的时候,
% 受到殷仲堪和桓玄的追拿,
% 奔逃到豫章,
% 生死未卜。
% 王茂和的儿子王彦猷
% 登时在京都,
% 终日担忧哀愁,
% 甚至已经按服孝的规矩
% 把衣食住行的规格降了下来。
% 当时的人们都说他是「见习孝子」。

% %% 妖
% 王愉任江州刺史期间,
% 被殷、桓的军队驱逐了,
% 于是逃亡至豫章后,音讯无踪。
% 奔窜豫章,存亡未测。
% 这是他的儿子王绥还在首都,
% 每日愁眉不展,
% 甚至已经按照服孝的规矩来约束自己的衣食住行。
% 当时的人们都称赞他为“
% 试守孝子”。
