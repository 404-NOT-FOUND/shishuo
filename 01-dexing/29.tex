
\switchcolumn*[\section{}]

王长豫为人谨顺,事亲尽色养之孝%
\footnote{%
    王长豫:王悦,字长豫,是王导的长子,
            名望很高,能承欢膝下,得到王导的偏爱,
            官至中书侍郎。
    色养:指侍养父母有喜悦的容色。
}%
。
丞相见长豫辄喜,见敬豫辄嗔%
\footnote{%
    丞相:即王导。
    敬豫:王恬,字敬豫,是王导的次子,
          放纵好武,不拘礼法,
          曾任魏郡太守。
}%
。
长豫与丞相语,恒以慎密为端。
丞相还台,及行,未尝不送至车后%
\footnote{%
    台:中央机关的官署,这里指尚书省。
}%
。
恒与曹夫人并当箱箧%
\footnote{%
    曹夫人:王导的妻子。
    并当:也作屏当。整理;收拾。
    箧(qiè):小箱子。
}%
。
长豫亡后,丞相还台,登车后,哭至台门;
曹夫人作簏,封而不忍开%
\footnote{%
    簏(lù):竹箱子。
}%
。

%% ----------------------------------------------------------------------------
\switchcolumn

% %% Jy
% 王长豫为人谨慎和顺,
% 对父母非常孝顺。
% 他父亲王丞相见到他就喜笑颜开,
% 见到他弟敬豫就愁眉不展。
% 长豫和丞相说话的时候,
% 总是谨慎用辞,语言缜密。
% 丞相每次回尚书省,
% 长豫都要把父亲送到车前。
% 长豫还常常帮母亲曹夫人
% 一起收拾箱柜衣物。
% 长豫死后,
% 丞相每次回尚书省的时候
% 都一路哭到尚书省门前;
% 而曹夫人每次再收拾箱柜的时候,
% 也难过得迟迟不忍打开那些记忆。

% % %% 妖
% 王长豫为人孝顺谨慎,侍奉父母神色愉悦,尽心尽力。
% 他爹王丞相见到他就很开心,而见到他弟弟王敬豫就生气。
% 长豫和父亲说话时,总是尽量小心周密。
% 王丞相去尚书省时,长豫每次都送他上车。
% 他也常常帮他母亲曹夫人收拾箱笼。
% 王长豫死后,他爹再去尚书省时,从上车哭到下车。
% 他娘收拾东西时,把他曾经装好的箱子封起来,不忍打开。
