
\switchcolumn*[\section{}]

殷仲堪既为荆州%
\footnote{%
    殷仲堪:表字不详,
            笃信天师道,生活俭省,可是事神不借钱财。
}%
,
值水俭%
\footnote{%
    水俭:因水灾而年成不好。
    俭:歉收。
}%
,
食常五碗,外无余肴,
饭粒脱落盘席间,辄拾以啖之%
\footnote{%
    啖(dàn):吃。
}%
。
虽欲率物,亦缘其性真素
。
每语子弟云:「
    勿以我受任方州,
    云我豁平昔时意%
    \footnote{%
        豁(huò):抛弃。
        时意:时俗。
    }%
    ,
    今吾处之不易。
    贫者,士之常,
    焉得登枝而捐其本?
    尔曹其存之%
    \footnote{%
        曹:等,辈;尔曹,你们。
        其:表命令、劝告的语气副词。
    }%
    !
」

%% ----------------------------------------------------------------------------
\switchcolumn

% %% Jy
% 殷仲堪任荆州刺史的时候,
% 正值涝灾,
% 饭都吃得不多,
% 也不加餐。
% 有饭粒掉到盘子里、座席上,
% 都要捡起来吃掉。
% 他这么做,
% 一方面是想树一个好的榜样,
% 亦一方面也是因为他生性就非常俭朴。
% 他常常教导晚辈们说:「
%     别因为我现在做着刺史,
%     就以为我要扔下了平常的作风,
%     到现在我也没有丝毫改变。
%     贫困呢,是我们读书人的常态,
%     不能说你登上树梢变了凤凰了,
%     就忘了本了。
%     你们都要牢牢地记住这一点。
% 」

% %% 妖
% 殷仲堪任荆州刺史时,
% 正遇水祸,辖内歉收,
% 每天饭吃的不多,也从不加菜,
% 当有饭粒掉到桌子上,就捡起来吃掉。
% 食常五碗,外无余肴,
% 诚然一方面是为了装出表率,但也是因为他生性简朴。
% 他常常教育家中晚辈:“
%  不要因为我做了刺史,
%  就说我抛弃了朴素的习惯,
%  我坐在如今的位置上也是战战兢兢的。
%  读书人,就是要安贫乐道,
%  哪里能因为做了官就忘了本呢?
%  你们都要牢牢记住这一点。”
