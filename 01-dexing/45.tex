
\switchcolumn*[\section{}]

吴郡陈遗,
家至孝%
\footnote{%
    陈遗:人名,资料不详。
}%
。
母好食铛底焦饭,
遗作郡主簿,
恒装一囊,
每煮食,
辄伫录焦饭,
归以遗母%
\footnote{%
    铛(chēng):一种铁锅。
    贮录:贮藏。
}%
。
后
值孙恩贼出吴郡,
袁府郡即日便征%
\footnote{%
    孙恩:字灵秀。
          东晋末,孙恩聚众数万,攻陷郡县。
          后来攻打临海郡时被打败,
          跳海身亡。
    袁府君:即袁山松,又说袁崧,字不详。
            博学有文章,善音乐。
            孙恩作乱时,袁山松任吴郡太守,
            筑守沪渎垒与之对抗,
            城陷被杀。
}%
。
遗已聚敛得数斗焦饭,
未展归家,
遂带以从军%
\footnote{%
    未展:未及。
}%
。
战于沪渎,
败。
军人溃散,
逃走山泽,
皆多饥死,
遗独以焦饭得活。
时人以为纯孝之报也。

%% ----------------------------------------------------------------------------
\switchcolumn

% %% Jy

% %% 妖

