
\switchcolumn*[\section{}]

吴郡陈遗,
家至孝%
\footnote{%
    陈遗:人名,资料不详。
}%
。
母好食铛底焦饭,
遗作郡主簿,
恒装一囊,
每煮食,
辄伫录焦饭,
归以遗母%
\footnote{%
    铛(chēng):一种铁锅。
    主簿:各级主官属下掌管文书的佐吏。
          魏、晋以前主簿官职
          广泛存在于各级官署中;
          隋、唐以后,
          主簿是部分官署与地方政府的事务官,
          重要性减少。
    贮录:贮藏。
}%
。
后
值孙恩贼出吴郡,
袁府郡即日便征%
\footnote{%
    孙恩:字灵秀。
          东晋末,孙恩聚众数万,攻陷郡县。
          后来攻打临海郡时被打败,
          跳海身亡。
    袁府君:即袁山松,又说袁崧,字不详。
            博学有文章,善音乐。
            孙恩作乱时,袁山松任吴郡太守,
            筑守沪渎垒与之对抗,
            城陷被杀。
}%
。
遗已聚敛得数斗焦饭,
未展归家,
遂带以从军%
\footnote{%
    未展:未及。
}%
。
战于沪渎,
败。
军人溃散,
逃走山泽,
皆多饥死,
遗独以焦饭得活。
时人以为纯孝之报也。

%% ----------------------------------------------------------------------------
\switchcolumn

% %% Jy
% 吴郡有一个叫陈遗的孝子。
% 陈遗的母亲喜欢吃锅巴,
% 陈遗那时候任郡中的主簿,
% 常常带上一个口袋,
% 一到煮饭的时候,
% 就刮一点锅巴
% 带回家给母亲吃。
% 后来
% 孙恩进犯吴郡的时候,
% 吴郡太守袁山松
% 马上展开防守。
% 陈遗那时候刚装下几斗的锅巴,
% 还没等下班回家,
% 就跟着出征了。
% 孙恩军和袁军在沪渎大战,
% 袁军不敌,
% 军队溃散开来,
% 四处奔逃,
% 大部分人最后都饿死了。
% 只有陈遗靠着自己装下的锅巴
% 得以存活。
% 人们都说
% 这是他一片孝心
% 得到了福报。

% %% 妖

