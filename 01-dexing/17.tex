
\switchcolumn*[\section{}]

王戎、和峤同时遭大丧,具以孝称。
王鸡骨支床,和哭泣备礼%
\footnote{%
    按:王戎曾任光禄勋、吏部尚书,因母亲丧事离职。
        服丧期间,不拘礼制,饮酒食肉,但面容憔悴。
    和峤(qiáo):字长舆,任中书令、尚书,因母亲丧事离职。
                  服丧期间,谨守礼法,量米而食,不多吃饭,但不如王戎憔悴。
    大丧:父母之丧。
    鸡骨支床:指骨瘦如柴。
}%
。
武帝谓刘仲雄曰%
\footnote{%
    刘仲雄:名毅,字仲雄,为人刚直,任司隶校尉、尚书左仆射。
}%
:「
    卿数省王、和不%
    \footnote{%
        卿:君称臣为卿。
        数(shuò):屡次;经常。
        省(xǐng):探望。
        不(fǒu):同否。
    }%
    ?
    闻和哀苦过礼,使人忧之。
」
仲雄曰:「
    和峤虽备礼,神气不损;
    王戎虽不备礼,而哀毁骨立%
    \footnote{%
        哀毁骨立:形容悲哀过度,瘦弱不堪,只剩个骨架立着。
    }%
    。
    臣以和峤生孝,王戎死孝%
    \footnote{%
        生孝:指遵守丧礼而能注意不伤身体的孝行。
        生孝:指对父母尽哀悼之情而至于死的孝行。
    }%
    。
    陛下不应忧峤,而应忧戎。
」

%% ----------------------------------------------------------------------------
\switchcolumn

% %% Jy
% 王戎与和峤同时丧母,都因为尽孝的行为而出名。
% 王戎当时骨瘦如柴,和峤虽痛哭哀悼,但礼仪周到。
% 武帝对刘仲雄说:「
%     你是不是常常去探望王、和二人?
%     我听说和峤过于伤痛,超出了礼法常规,真令人担忧啊。
% 」
% 仲雄说:「
%     和峤虽然礼仪周到,但精神还好;
%     王戎虽然礼仪不周,可是弄垮了身体。
%     臣认为和峤行的是生孝,王戎行的却是死孝。
%     陛下应该担心的,不是和峤,而应该是王戎啊。
% 」

% %% 妖
王戎与和峤同时丧母,他俩都以孝行而著称。
王戎已经骨瘦如柴,和峤痛哭流涕,礼数周至、
晋武帝对刘仲雄说:“
 你经常去探望王戎、和峤嘛?
 我听说他俩已经哀痛欲绝超出了礼法常规,这可真令人担忧啊。”
仲雄回答说:“
 和峤虽然礼数周全,但精气神都没有损伤;
 而王戎即使礼节不整,却好似行尸走肉。
 我以为和峤是生孝,王戎确是死孝。
 陛下您不该担心和峤,而应该担心王戎啊。”
