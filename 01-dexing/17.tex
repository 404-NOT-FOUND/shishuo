
\switchcolumn*[\section{}]

王戎、和峤同时遭大丧,具以孝称%
\footnote{%
    按:王戎曾任光禄勋、吏部尚书,因母亲丧事离职。
        服丧期间,不拘礼制,饮酒食肉,但面容憔悴。
    和峤(qiáo):字长舆,任中书令、尚书,因母亲丧事离职。
                  服丧期间,谨守礼法,量米而食,不多吃饭,但不如王戎憔悴。
    大丧:父母之丧。
}%
。
王鸡骨支床,和哭泣备礼%
\footnote{%
    鸡骨支床:指骨瘦如柴。
}%
。
武帝谓刘仲雄曰%
\footnote{%
    刘仲雄:名毅,字仲雄,为人刚直,任司隶校尉、尚书左仆射。
}%
:「
    卿数省王、和不%
    \footnote{%
        卿:君称臣为卿。
        数(shuò):屡次;经常。
        省(xǐng):探望。
        不(fǒu):同否。
    }%
    ?
    闻和哀苦过礼,使人忧之。
」
仲雄曰:「
    和峤虽备礼,神气不损;
    王戎虽不备礼,而哀毁骨立%
    \footnote{%
        哀毁骨立:形容悲哀过度,瘦弱不堪,只剩个骨架立着。
    }%
    。
    臣以和峤生孝,王戎死孝%
    \footnote{%
        生孝:指遵守丧礼而能注意不伤身体的孝行。
        生孝:指对父母尽哀悼之情而至于死的孝行。
    }%
    。
    陛下不应忧峤,而应忧戎。
」

%% ----------------------------------------------------------------------------
\switchcolumn

% %% Jy

% %% 妖
