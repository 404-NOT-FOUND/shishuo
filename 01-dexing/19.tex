
\switchcolumn*[\section{}]

王戎云:「
    太保居在正始中,不在能言之流%
    \footnote{%
        太保:指王祥。王祥曾任太保之职,这里以官名代人名。
        正始:三国时魏帝曹芳年号。
        能言:指能清谈。
              魏晋时士大夫崇尚清谈,
              主张不务实际,专谈玄理,这形成了一种风气。
    }%
    。
    及与之言,理中清远,将无以德掩其言%
    \footnote{%
        理中:恰当的义理;正理。
        按:《晋书·王祥传)作「理致」(义理和情致)。
        将无:莫非;恐怕,用来表示猜测而意思偏于肯定。
        按:《晋阳秋》中提到:「祥少有美德行。」〉
    }%
    。
」

%% ----------------------------------------------------------------------------
\switchcolumn

% %% Jy
% 王戎说:「
%     王祥王太保在正始年间,
%     并不算是清谈中的好手。
%     但实际与他交谈起来就会发现,
%     他言语清新道理深远,
%     恐怕是他的美德盖没了他清谈的能力吧。
% 」

% %% 妖
% 王戎说:“
%  太保是正始年间的人,不在能以言谈著称的人之列。
%  但等到与他谈话时,却发现其话语义理深远,大约他的德行比言语更著名吧。”
