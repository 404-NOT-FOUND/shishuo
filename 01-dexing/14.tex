
\switchcolumn*[\section{}]

王祥事后母朱夫人甚谨%
\footnote{%
    王祥:字休征,魏晋时人,是个孝子。
          因为侍奉后母,年纪很大才进入仕途,官至太常、太保。
}%
。
家有一李树,结子殊好,母恒使守之。
时风雨忽至,祥抱树而泣。
祥尝在别床眠,母自往暗斫之%
\footnote{%
    斫(zhuó):大锄;引申为用刀、斧等砍。
}%
。
值祥私起%
\footnote{%
    私:小便。
}%
,空斫得被。
既还,知母憾之不已,因跪前请死。
母于是感悟,爱之如己子。

%% ----------------------------------------------------------------------------
\switchcolumn

% %% Jy
% 王祥总是小心周谨地侍奉他的继母朱夫人。
% 他们家有一棵李树,结的果子都很好,
% 继母总让王祥看着。
% 有时刮风下雨,王祥就抱住树大哭。
% 有一次,王祥在别床睡,
% 继母去暗杀他。
% 正巧王祥起夜,
% 继母只砍到了一床空被。
% 王祥回来之后,
% 心知继母非常憾恨,
% 于是就跪在继母身前
% 请求一死。
% 继母深受感动,
% 从这以后把王祥视如己出。

% %% 妖
% 王祥侍奉他的后母朱夫人非常尽心。
% 他家里有一颗结果非常丰硕的李子树,朱夫人就吩咐他一直照看它。
% 有时刮起狂风下起暴雨,王祥就抱着李子树哭泣。
% 一次王祥在另一张床上睡觉,朱夫人就亲自偷偷地去砍杀他。
% 恰值他起夜,后母只砍到了被子。
% 等回来之后,发现朱夫人为此遗憾不已,就跪到她面前请求处死自己。
% 后母因此非常感动,醒悟过来,像对待亲生儿子一样爱他
