
\switchcolumn*[\section{}]

王恭从会稽还,
王大看之%
\footnote{%
    王恭:字孝伯,为官清廉,
          晋安帝时起兵反对帝室,
          被杀。
    会稽:郡名,郡治在今浙江省绍兴县。
    王大:王忱,小名佛大,也称阿大,
          是王恭的同族叔父辈。
          官至荆州刺史。
}%
。
见其坐六尺簟%
\footnote{%
    蕈(diàn):竹席。
}%
,
因语恭:「
    卿东来,
    故应有此物,
    可以一领及我。
」
恭无言。
大去后,
即举所坐者
送之。
既无余席,便坐荐上%
\footnote{%
    荐:草席。
}%
。
后大闻之,
甚惊,曰:「
    吾本谓卿多,故求耳。
」
对曰:「
    丈人不悉恭,
    恭作人无长物%
    \footnote{%
        长(zhàng)物:多余的东西。
    }%
    。
」

%% ----------------------------------------------------------------------------
\switchcolumn

% %% Jy
% 王恭从会稽回来,
% 他的叔叔王大去看望他。
% 王大看到他坐着一领六尺竹席,
% 于是就对王恭说:「
%     你从东边回来,
%     难怪有这样的好东西。
%     要不你拿一领给我吧?
% 」
% 恭没说什么。
% 等到王大回去了,
% 王恭便把他坐的那张席子
% 送给了王大。
% 这么一来,
% 自己没有席子坐了,
% 于是就坐到草席上。
% 后来王大听说了这件事,
% 非常震惊。
% 他告诉王恭
% 他本以为王恭有很多那样的席子
% 才和他要的。
% 王恭回答他说:「
%     叔叔你不了解我,
%     我这个人
%     不会去积攒多余的东西。
% 」

% %% 妖
% 王恭从会稽回来后,
% 王大去看望他。
% 发现王恭坐在一张六尺的竹席子上,
% 就对他说:“
%  您从东边来,
%  应该带回来挺多这种席子的吧,
%  所以给我一张吧。”
% 王恭没有回答。
% 王大走后,
% 王恭就把他所坐的那张竹席送给了他。
% 因为王恭并没有多余的席,就自己坐在了草席之上。
% 后来王大听说了这件事,
% 非常惊讶,说:“
%  我本来以为您有多余的,所以才向您讨要。”
% 王恭回答说:“
%  您不了解我,
%  我这个人从来没有多余的东西。”
