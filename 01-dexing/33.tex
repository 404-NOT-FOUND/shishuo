
\switchcolumn*[\section{}]

谢奕作剡令%
\footnote{%
    谢奕:字无奕,东晋名臣谢安的哥哥,年少时便气宇不凡。
    令:指县令,一县的行政长官。
}%
,
有一老翁犯法,
谢以醇酒罚之,乃至过醉,而尤未已。
太傅时年七八岁,着青布绔,在兄膝边坐%
\footnote{%
    太傅:指谢安,字安石,为人温雅畅达,学行俱佳。
    绔:即裤。
}%
,谏曰:「
    阿兄,老翁可念,何可作此%
    \footnote{%
        念:怜悯;同情。
    }%
    !
」
奕于是改容曰:「
    阿奴欲放去邪%
    \footnote{%
        阿奴:长对幼、尊对卑的称呼。
    }%
    ?
」
遂遣之。

%% ----------------------------------------------------------------------------
\switchcolumn

% %% Jy
% 谢奕任剡县令的时候,
% 有一位老人犯了法,
% 谢奕就罚他喝烈酒,
% 老头已经醉得不成样子了,
% 谢奕还没有罢休。
% 谢安石谢太傅当时才七八岁,
% 穿着青布套裤,
% 正坐在哥哥谢奕的膝上。
% 这时候他劝说谢奕道:「
%     哥哥,
%     老爷爷那么可怜,
%     怎么能这样对待他呢!
% 」
% 谢奕神色缓和下来,说:「
%     阿弟是想要放他走咯?
% 」
% 于是便放走了老人。

% %% 妖
% 谢奕做剡县县令时,
% 有一个老头犯了法,
% 谢奕就惩罚他喝醇酒,哪怕者老头已经醉得很厉害了,但还是没有停下灌酒。
% 谢安当时不过七八岁,还穿着蓝色的裤子,坐在他哥哥膝上,劝道:“
%  哥哥,您看这老人家多可怜啊,何苦这样为难他呢!”
% 谢奕的脸色和缓下来,说:“
%  你是想放他走嘛?”
% 于是就打发他离开了。
