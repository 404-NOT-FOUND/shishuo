
\switchcolumn*[\section{}]

荀巨伯远看友人疾%
\footnote{%
    荀巨伯:东汉人。
}%
,
值胡贼攻郡%
\footnote{%
    胡:古时西方、北方各少数民族统称胡。
}%
,
友人语巨伯曰:「
    吾今死矣,子可去%
    \footnote{%
        子:对对方的尊称,相当于「您」。
    }%
    !
」
巨伯曰:「
    远来相视,子令吾去;
    败义以求生,岂荀巨伯所行\mbox{邪%
    \footnote{%
        邪:文言疑问词,相当于「呢」或「吗」。
    }}%
    !
」
贼既至,谓巨伯曰:「
    大军至,一郡尽空,汝何男子,而敢独止?
」
巨伯曰:「
    友人有疾,不忍委之%
    \footnote{%
        委:舍弃,抛弃。
    }%
    ,宁以吾身代友人命。
」
贼相谓曰:「
    吾辈无义之人,而入有义之国。
」
遂班军而还,一郡并获全%
\footnote{%
    班军:班师;调回出征的军队。
}%
。

%% ----------------------------------------------------------------------------
\switchcolumn

荀巨伯远道来探望罹病的友人,
正碰见胡人入犯,攻打城郡。
朋友对巨伯说:「
    看来我是大期将至了。您快先逃走吧!
」
巨伯说道:「
    % 我远道而来,就为了看看您的病情,
    我远道来看你,
    您却叫我独自逃生;
    我荀巨伯能是这样不仁不义、苟且偷生的人吗!
」
胡军打到郡内,喝问巨伯:「
    我们大军一路杀进来,
    郡里的人早逃得一干二净。
    你是何方神圣,
    居然敢留下来送死?
」
巨伯道:「
    我朋友身患疾病,走不了;
    我不忍心弃他而去,
    愿一命换一命,
    保他一条生路。
」
胡人面面相觑,说道:「
    看来我们是一帮无情无义之人
    闯入了这有情有义的国度呀。
」
随即回调大军,返回了家乡。
这个城郡也因此得以保全。


