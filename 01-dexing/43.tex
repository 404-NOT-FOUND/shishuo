
\switchcolumn*[\section{}]

桓南郡既破殷荆州%
\footnote{%
    以王恭为首的起义失败后,
    殷、桓等都各回领地。
    桓与杨佺期有矛盾,
    桓玄于是攻据荆州,
    杀杨佺期,
    后
    逼死殷仲堪。
}%
,
收殷将佐十许人,
咨议罗企生亦在焉%
\footnote{%
    罗企生:字宗伯,
            桓玄进犯荆州的时候,
            为殷仲堪幕府任咨议参军,掌管谋划。
            殷仲堪多疑少决,
            罗企生对弟弟遵生说:「
                殷侯仁而无断,事必无成。
                成败
                天也,
                吾当死生以之。
            」
            殷奔走的时候,
            没有官员愿意跟随,
            只有企生还跟着殷。
            经过罗家,
            遵生假说生离死别,要与哥哥握手,
            趁机将企生拉下马,
            说
            「家有老母,将欲何行?」
            企生挥泪说
            自己必死无疑,
            让弟弟好好地照顾母亲,
            「一门之内,有忠与孝,亦复何恨!」
            兄弟更难分舍。
            企生请等待的殷仲堪稍候,
            仲堪看他们兄弟难分,
            策马而去。
            等桓玄占据了荆州,
            百官都去谒见桓玄,
            只有企生不愿去。
            别人都说
            这样桓玄一定不会放过他;
            企生说:「
                我殷侯吏,
                见遇以国士,
                不能共殄丑逆,
                致此奔败,
                何面目
                就桓求生乎?
            」
            桓玄听说后大怒,
            说他对待企生不薄,
            企生怎么不领情,
            企生说:「
                使君口血未干
                (
                    指殷、桓等人
                    联盟时喝的血酒
                ),
                而生此奸计。
                自伤力劣,
                不能
                剪定凶逆,
                我死
                恨晚尔!
            」
            桓玄于是将他斩死。
}%
。
桓素待企生厚,
将有所戮,先遣人语云:「
    若谢我,
    当释罪。
」
企生答曰:「
    为殷荆州吏,
    今荆州奔亡,存亡未判,
    我何颜谢桓公?
」
既出市%
\footnote{%
    市:刑场。
}%
,
桓又遣人问:「
    欲何言?
」
答曰:「
    昔
    晋文王杀嵇康,
    而嵇绍为晋忠臣%
    \footnote{%
        嵇绍:嵇康的儿子。
              嵇康被司马昭诬害处死,
              但嵇绍在晋代累升至散骑常侍。
              晋惠帝亲征成都王司马颖时,
              败于荡阴,
              百官逃散,独嵇绍以身保卫惠帝而死。
    }%
    。
    从公乞一弟
    以养老母。
」
桓亦如言宥之%
\footnote{%
    宥(yòu):宽容,饶恕。
}%
。
桓先曾以一羔裘与企生母胡,
胡时在豫章,
企生问至%
\footnote{%
    问:消息。
}%
,即日焚裘。

%% ----------------------------------------------------------------------------
\switchcolumn

% %% Jy
% 桓玄攻破荆州、
% 逼死殷仲堪之后,
% 收取了殷仲堪的旧部下十来个人。
% 这时候,
% 桓玄就打听
% 罗企生是不是已经来了。
% 桓玄对待罗企生非常优厚,
% 准备要斩杀罗企生之前,
% 特意派人传话给罗企生说:「
%     你只要向我谢罪,
%     我就放过你。
% 」
% 企生回复说:「
%     我是殷刺史的手下,
%     现在殷刺史奔逃在外,
%     生死不明,
%     我还向桓公谢罪?
%     那我还有什么脸面做人呢!
% 」
% 等到他上了法场,
% 桓玄又让人问他还有什么话说,
% 罗企生答道:「
%     旧时晋文王杀了嵇康,
%     但是嵇绍却是晋国的大忠臣。
%     我请桓公放过我的弟弟,
%     让他能赡养我的老母亲。
% 」
% 桓玄果然照他说的,放过了罗企生的弟弟。
% 先前
% 桓玄曾经送了一领羊皮大衣给企生的母亲胡氏。
% 企生遇害后,
% 遥在豫章的胡氏听说了他的噩耗,
% 当下把那羊皮大衣拿出来
% 一把火烧了。

% %% 妖
% 桓玄打败荆州刺史殷仲堪之后,
% 俘虏了他的幕僚十几人,
% 咨议参军罗企生也在其中。
% 桓素对待罗企生很好,
% 在准备大开杀戒前,事先派人对他说:“
%  你要是向我认罪,我就会放你一马。”
% 罗企生回答说;“
%  我是殷大人的属下,
%  现在主公逃亡在外,生死未卜,
%  这样就向桓公谢罪,我哪有脸面做人呢?”
% 把他押赴刑场之后,
% 桓玄又派人询问他:“
%  你有什么遗言要说吗?”
% 罗企生说:“
%  当年晋文王杀了嵇康,
%  但他的儿子嵇绍仍然为国尽忠。
%  因此我想向桓公祈求,留我弟弟一命来侍奉母亲。”
% 桓玄就如他所愿,宽恕了他弟弟。
% 之前,桓玄送给胡企生母亲胡氏一件羊羔皮的大衣。
% 当时胡氏远在豫章,
% 儿子遇害的消息一传来,
% 她立马就把大衣扔到火里烧了。
