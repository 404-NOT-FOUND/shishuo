
\switchcolumn*[\section{}]

李元礼风格秀整,高自标持,欲以天下名教是非为己任%
\footnote{%
    李元礼:
            名膺(yīng),字元礼,东汉人,曾任司隶校尉。
            当时朝廷纲纪废弛,他却独持法度,以声名自高。
            后谋诛宦官未成,被杀。
    风格秀整:风度出众。品性端庄。
    高自标持:自视甚高;很自负。
    名教:以儒家所主张的正名定分为准则的礼教。
}%。
后进之士,有升其堂者%
\footnote{%
    升其堂:登上他的厅堂,指有机会接受教诲。
}%
,皆以为登龙门%
\footnote{%
    龙门:在山西省河津县西北。
          那里水位落差很大,传说龟鱼不能逆水而上,
          有能游上去的,就会变成龙。
}%
。

%% ----------------------------------------------------------------------------
\switchcolumn

李元礼风度出众,品性端庄,自视甚高。
他把在全国推行儒家礼教、辨明是非看成自己的责任。
后辈的读书人,如果有能进入他家聆听教诲的,都认为自己像是登上了龙门。

