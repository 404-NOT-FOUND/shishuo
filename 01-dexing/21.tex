
\switchcolumn*[\section{}]

王戎父浑,
有令名,官至凉州刺史%
\footnote{%
    王浑:字长源,有才望。
    令:好的,美好的。
    刺史:晋代全国分若干个州,州的最高行政长官称刺史。
}%。
浑薨%
\footnote{%
    薨(hōng):古代王侯去世叫做薨。
                王浑曾被封为贞陵亭侯,所以他的死可以称薨。
}%
,
所历九郡义故,怀其德惠%
\footnote{%
    九郡:据《晋书·地理志),凉州管辖八个郡,所以有以为这里的九郡应是八郡。
          但是也有说《御览)是引作「州郡」的,认为「九」是「州」的误字。
    义故:义从和故吏。
          指自愿受私人招募从军的官佐和过去的部下。
}%
,
相率致赙数百万,戎悉不受%
\footnote{%
    赙(fù):送给别人办丧事的财物。
}%
。

%% ----------------------------------------------------------------------------
\switchcolumn

% %% Jy
% 王戎的父亲王浑
% 享有美名,
% 官位也做到凉州剌史。
% 王浑过世时,
% 曾经跟随他、在他手下做事的人们都非常感怀他的恩德,
% 一起筹集了数百万的财物,
% 献到王家办丧事,
% 王戎什么也没有收下%
% \footnote{%
%     按:虞预《晋书》评说「戎由是显名。」
% }%。

% %% 妖
% 王戎的父亲王浑,
% 名望很大,并且做官做到了凉州刺史。
% 王浑过世后,其所辖九郡的门生旧吏,都感念他的功德与恩惠,
% 大伙相继凑了数百万钱作为丧葬费,而王戎分文未收。
