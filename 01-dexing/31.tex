
\switchcolumn*[\section{}]

庾公乘马有的卢%
\footnote{%
    庾公:庚亮,字元规,任征西大将军、荆州刺史。
    的卢:又作的颅,是额上有白色斑点的马,古人认为这是凶马,它的主人会得祸。
}%
,
或语令卖去,
庾云:「
    卖之必有买者,即当害其主,宁可不安己而移于他人哉?
    昔
    孙叔敖杀两头蛇以为后人,古之美谈%
    \footnote{%
        孙叔敖:春秋时代楚国的令尹。
                据贾谊《新书》载,
                孙叔敖小时候在路上看见一条两头蛇,回家哭着对母亲说:
                「听说看见两头蛇的人一定会死,我今天竟看见了。」
                母亲问他蛇在哪里,
                孙叔敖说:
                「我怕后面的人再见到它,就把它打死埋掉了。」
                他母亲说:
                「你心肠好,一定会好心得好报,不用担心。」
    }%
    。
    效之,不亦达乎?
」

%% ----------------------------------------------------------------------------
\switchcolumn

% %% Jy
% 庾公有一匹的卢马,
% 传说的卢是凶马,
% 有人便建议他将马卖去。
% 庾公说:「
%     我把它卖给谁,
%     那谁不就得遭殃了?
%     怎么能说
%     因为它会克我,
%     我就把灾祸扔给别人呢?
%     古时候孙叔敖为了后来人着想而将双头蛇杀死的故事
%     历来被人称道。
%     今天我只有效仿他,
%     才能算是明事达理。
% 」

% %% 妖

