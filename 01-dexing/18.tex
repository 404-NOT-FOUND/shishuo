
\switchcolumn*[\section{}]

梁王、赵王,国之近属,贵重当时%
\footnote{%
    梁王:司马肜(róng),司马懿的儿子。
          晋武帝(司马懿的孙子)即位后,封梁王,后任征西大将军,官至太宰。
    赵王:司马伦,司马懿的儿子。晋武帝时封赵王,晋惠帝时起兵反,自为相国,又称皇帝,后败死。
}%
。
裴令公岁请二国租钱数百万,以恤中表之贫者%
\footnote{%
    裴令公:裴楷,字叔则,官至中书令,尊称为裴令公。
    二国:指梁王、赵王两人的封国。国是侯王的封地。
    恤:周济。
    中表:指中表亲,跟父亲的姐妹的子女和母亲的兄弟姐妹的子女之间的亲戚关系。
}%
。
或讥之曰:「何以乞物行惠?」
裴曰:「损有余,补不足,天之道也 。」

%% ----------------------------------------------------------------------------
\switchcolumn

% %% Jy
梁王与赵王都是皇室子弟,是身份显赫的人。
裴令公每年都向两位王爷借金数百万,
来周济家中有困难的人。
有的人就嘲讽他说:「
    向人讨钱,再拿来施舍,
    这叫什么道理?
」
裴公说:「
    \removed{拆东墙补西墙}
    挪用富余的资源来补助不足的资源,
    这事儿天经地义。
」

% %% 妖

