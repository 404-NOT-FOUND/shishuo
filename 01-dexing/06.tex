
\switchcolumn*[\section{}]

陈太丘诣荀朗陵%
\footnote{%
    陈太丘:名寔,字仲弓,曾任太丘县长,所以称陈太丘。古代常以官名称人。
    荀朗陵:指荀淑(见\ref{sec:李元礼赞贤})。
}%
,贫俭无仆役,
乃使元方将车,季方持杖后从%
\footnote{%
    元方、季方:都是陈寔的儿子。
                元方是长子,名纪,字元方;
                季方是少子,名湛,字季方。
                父子三人才德兼备,知名于时。
}%
。
长文尚小,载着车中%
\footnote{%
    长文:陈寔的孙子陈群。
}%。
既至,荀使叔慈应门,慈明行酒,余六龙下食,文若亦小,坐着膝前%
\footnote{%
    叔慈、慈明、六龙:苟淑有八个儿子,号称八龙。
                      叔慈、慈明是他两个儿子的名字,
                      其余六人就是这里所说的六龙了。
    文若:荀淑的孙子荀或。
    下食:上莱。
    膝前:膝上。
}%。
于时太史奏 :「真人东行%
\footnote{%
    太史:官名,主要掌管天文历法。
    真人:修真得道的人。此应指某星。
}%
。」

%% ----------------------------------------------------------------------------
\switchcolumn

太丘县长陈寔去拜访朗陵侯荀淑。
因为家中清苦节俭,没有仆役侍候,
陈寔就让长子元方驾车送他,少子季方拿着手杖跟在车后。
孙子长文年纪还小,就一起坐在车上。
到了荀家,
荀淑让叔慈迎接客人,让慈明敬酒,其余六个儿子管上菜。
孙子文若也还小,就坐在荀淑膝上。
与此同时,太史启奏朝廷说:「星相发生变化,真人星向东移去了%
\footnote{%
    古时人们认为星相变化和世上的重大事件有关。
    作者此处意在说明陈荀两家贤人聚首的事件引发了真人星的星相变化。
}%
。」

