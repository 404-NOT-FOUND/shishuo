
\switchcolumn*[\section{}]

郭林宗至汝南,造袁奉高%
\footnote{%
    郭林宗:名泰,字林宗,东汉人,博学有德,为时人所重。
    袁奉高:名阆(làng),字奉高,和黄叔度同为汝南郡慎阳人;
            多次辞谢官府任命,也很有名望。
            曾为汝南郡功曹,后为太尉属官。
}%
,车不停轨,鸾不辍轭%
\footnote{%
    轨:车子两轮之间的距离,其宽度为古制八尺,后引申为车辙。
    辍(chuò):停止。
    鸾:装饰在车上的铃子,这里指车子。
    轭(è):驾车时搁在牛马颈上的曲木。
}%
;
诣黄叔度,乃弥日信宿%
\footnote{%
    弥日:终日;整天。
    信宿:连宿两夜。
}%。
人问其故,
林宗曰:「
    叔度汪汪如万顷之陂,澄之不清,扰之不浊%
    \footnote{%
        陂(bēi):湖泊。
    }%
    ,其器深广,难测量也。
」

%% ----------------------------------------------------------------------------
\switchcolumn

郭林宗到汝南郡
拜访袁奉高时,刚见面一会儿就走了,好像根本没停下车来似的;
待到他去拜访黄叔度,却一住就是两天。
别人问他缘故,
他说:「
    黄叔度就好比万顷的湖泊一般,澄不清也搅不浑;
    他深广的气量,着实难以测量!
」

