
\switchcolumn[0]*[\section{}]

蔡洪赴洛,
洛中人问曰:「
    幕府初开,
    群公辟命,
    求英奇于仄陋,
    采贤俊于岩穴。
    君吴、楚之士,
    亡国之余,
    有何异才
    而应斯举?
」
蔡答曰:「
    夜光之珠,不必出于孟津之河;
    盈握之璧,不必采于昆仑之山。
    大禹生于东夷,
    文王生于西羌。
    圣贤所出,何必常处。
    昔武王伐纣,
    迁顽民于洛邑,
    得无诸君是其苗裔乎?
」

%% ----------------------------------------------------------------------------
\switchcolumn

% %% Jy

% %% 妖
% 蔡洪来到洛阳,
% 洛阳人问他:“
%  官署刚刚成立,
%  百官都在招贤纳士,
%  在草莽寒舍中寻找英奇之才,
%  于荒郊野岭处寻找贤俊之人,
%  您来自于吴楚之地,
%  乃是亡国丧家之犬,
%  有什么特殊才能感来此处应征?”
% 蔡洪回答说:“
%  夜明珠不一定出产于黄河,
%  手掌握不下的碧玉也不一定开采于昆仑山脉,
%  大禹在东夷之地出生,
%  周文王也来自于西羌,
%  圣贤诞生之处,哪里有固定的地方呢,
%  武王伐纣的时候,
%  将冥顽不化的商朝百姓迁到了洛邑,
%  莫非诸位就是那些刁民的后代嘛?”
