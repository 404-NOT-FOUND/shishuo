
\switchcolumn[0]*[\section{}]

蔡洪赴洛%
\footnote{%
    蔡洪:
        字叔开,吴郡人,
        原在吴国做官,吴亡后入晋,
        公认才华出众,
        西晋初年太康年间,
        由本州举荐为秀才,
        到京都洛阳。
}%
,
洛中人问曰:「
    幕府初开,
    群公辟命,
    求英奇于仄陋,
    采贤俊于岩穴%
    \footnote{%
        幕府:
            原指将军的官署,也用来指军政大员的官署。
        群公:
            众公卿,指朝廷中的高级官员。
        辟命:
            征召。
        仄陋:指出身贫贱的人。
        岩穴:山中洞穴。
    }%
    。
    君吴、楚之士,
    亡国之余,
    有何异才
    而应斯举%
    \footnote{%
        吴楚:
            春秋时代的吴国和楚国。
            两国都在南方,所以也泛指南方。
    }%
    ?
」
蔡答曰:「
    夜光之珠,不必出于孟津之河;
    盈握之璧,不必采于昆仑之山%
    \footnote{%
        夜光之珠:
            即夜明珠,
            是春秋时代隋国国君的宝珠,又叫隋侯珠,或称隋珠,
            传说
            隋侯出行,
            遇到一条断为两戳的大蛇,
            隋侯把两戳连到了一起,
            大蛇于是得以活命。
            后来为了报答隋侯的恩情,
            从江中衔来了一颗明珠,
            便是夜明珠。
        孟津:
            渡口名,在今河南省盂县南。
            周武王伐纣时和各国诸侯在这里会盟,是一个有名的地方。
        盈握:
            满满一把。这里形容大小。
        壁:中间有孔的圆形玉器。
            史上最出名的玉璧和氏璧
            就出自市井。
            相传,
            楚人卞和在荆山上发现了一块石头,
            他认定其中包藏有美玉,
            于是进献楚厉王。
            但玉工认为这只是一块普通的石头,
            于是卞和被砍去了左脚。
            后来楚武王即位,
            卞和二献璧,
            又被砍去了右脚。
            直到后来楚文王即位,
            听说卞和抱石痛哭了三天三夜,
            眦崩血泪,
            问其究竟,
            卞和说脚没了就没了,
            但不能承受
            美玉不被赏识、
            忠臣当作逆子。
            文王于是令人打开岩石,
            这才看到了这块惊世的玉璧。
            为纪念卞和,
            文王于是将它命名为「和氏璧」。
        昆仑:
            古代盛产美玉的山。
    }%
    。
    大禹生于东夷,
    文王生于西羌%
    \footnote{%
        大禹:
            传说中的一个君主,
            建立了夏朝,
            曾治平洪水。
        东夷:
            我国东部的各少数民族。
            按《孟子》的记载,
            生在东夷的不是大禹,
            而是传说中另一个贤君舜,
            舜禅让给了禹。
        文王:
            周文王,
            殷商时一个诸侯国的国君,封地在今陕西一带。
        西羌:
            我国西部的一个民族。
    }%
    。
    圣贤所出,何必常处。
    昔武王伐纣,
    迁顽民于洛邑,
    得无诸君是其苗裔乎%
    \footnote{%
        「昔武王」句:
            周武王灭了殷纣以后,
            把殷朝的顽固人物迁到洛水边上,
            派周公修建洛邑安置他们。
            战国以后,洛邑改为洛阳。
        得无:
            莫非。
        按:
            据诸多史料记载,
            秀才华谭(字令思)曾经被
            王济(字武子)嘲笑
            而有雷同对话,
            因此许多人认为
            这里其实是把华谭的故事张冠李戴了。
    }%
    ?
」

%% ----------------------------------------------------------------------------
\switchcolumn

% %% Jy
% 蔡洪中举,
% 到京都洛阳来。
% 当地人问他:「
%     我们刚刚稳定下来,
%     正是广召人才的时候。
%     要
%     在市井乡田里寻觅
%     能人异士,
%     在山林岩穴中谐访
%     贤者英才;
%     你一个国破家亡的南方佬,
%     竟也敢来这里应聘,
%     你有什么特长吗?
% 」
% 蔡洪答道:「
%     夜光明珠,
%     不一定要到孟津之河里去捞取;
%     名贵玉璧,
%     不一定要到昆仑名山里去采集。
%     大禹多么伟大,
%     却是东夷出身;
%     文王何不贤明,
%     却是西羌后人。
%     圣贤哪里又是某些地方的特产呢!
%     旧时
%     武王伐纣,
%     把顽抗分子都迁居到了洛邑,
%     这才有了洛阳;
%     难道你们
%     也都自诩是他们的后代不成?
% 」

% %% 妖

