
\switchcolumn[0]*[\section{}]

中朝有小儿,
父病,
行乞药%
\footnote{%
    中朝:
        西晋晋帝室南渡后
        称渡江前的西晋为中朝。
}%
。
主人问病,
曰:「
    患疟也。
」
主人曰:「
    尊侯明德君子,
    何以病疟%
    \footnote{%
        尊侯:
            尊称对方的父亲。
        何以病疟:
            当时俗传行疟的是疟鬼,
            形体极小,
            不敢使大人物得病,
            所以药店主人这样问。
    }%
    ?
」
答曰:「
    来病君子,
    所以为疟耳。
」

%% ----------------------------------------------------------------------------
\switchcolumn

% %% Jy
% 中朝的时候
% 有一个小孩,
% 他的父亲生病了,
% 于是就出去求药。
% 药铺主人
% 一问之下,
% 听说是得了疟疾,
% 便问道:「
%     令尊是德高望重的君子,
%     怎么会招来疟鬼的侵扰呢?
% 」
% 小孩听了连忙说:「
%     是啊,
%     连堂堂君子都要祸害,
%     这还不「虐」吗!
% 」

% %% 妖
% 西晋时有个小孩,
% 他的父亲生病了,
% 就外出行乞讨药。
% 药铺主人问是什么病,
% 小孩回答:“
%  是疟疾。”
% 主人说:“
%  令尊乃是有德的君子,
%  怎么会患疟疾呢?”
% 小孩达到:“
%  正是因为这种病会来祸害君子,
%  才被成为疟疾啊。”
