
\switchcolumn[0]*[\section{}]

祢衡被魏武
谪为鼓吏%
\footnote{%
    祢(mí)衡:
        字正平,
        汉末建安时人,
        孔融曾向魏王曹操推荐他。
        曹操想接见,他不肯去见,
        而且有不满言论。
        相传
        曹操很生气,
        想羞辱他,
        便派他做鼓吏(击鼓的小吏)。
        各版史书记录不一,
        不可考。
    魏武:
        曹操,
        初封魏王,死后溢为武。
        其子曹丕登帝位建立魏国后,
        追尊为武帝。
    谪(zhé):降职。
}%
,
正月半试鼓,
衡扬枹为《渔阳掺挝》%
\footnote{%
    月半试鼓:
        《文士传》记载此事时说:
        「后至八月朝会,大阅试鼓节」。
    枹(fú):
        鼓槌。
    《渔阳掺挝(sān zhuā)》:
        鼓曲名,
        也作渔阳参挝。
        掺,通叁,即三;
        挝,鼓槌。
        三挝,指鼓曲的曲式为三段体,
        犹如古曲中有三弄、三叠之类。
        此曲为祢衡所创,
        取名渔阳,
        是借用东汉时彭宠据渔阳反汉的故事。
        彭宠据幽州渔阳反,攻陷蓟城,
        自立力燕王,
        后被手下的人杀死。
        祢衡击此鼓曲,
        有讽刺曹操反汉的意思。
}%
,
渊渊有金石声,
四坐为之改容%
\footnote{%
    渊渊:形容鼓声深沉。
    金石:指钟磬一类乐器。
}%
。
孔融曰:「
    祢衡罪同胥靡,
    不能发明王之\mbox{梦%
    \footnote{%
        胥(xū)靡:
            轻刑名,指服劳役的囚徒。
        明王之梦:
            商朝君主武丁梦见
            上天赐给他一个贤人,
            就令人画出其相貌去寻找,
            果然找到一个正在服劳役的囚徒,
            他就是后来成为商代贤相的傅说。
    }}%
    。
」
魏武惭而赦之。

%% ----------------------------------------------------------------------------
\switchcolumn

%% Jy

% %% 妖

