
\switchcolumn[0]*[\section{}]

佛图澄与诸石游,
林公曰%
\footnote{%
    佛图澄:
        和尚名,
        晋代永嘉年间到洛阳。
    诸石:
        指石勒(字世龙)、石虎(字季龙)等人,
        羯族人。
        东晋时石勒侵入中原,大肆杀戮,
        建立后赵政权。
        石勒死后,堂弟石虎杀石勒诸兄而袭位。
        石勒、石虎兄弟都很敬重佛图澄。
    林公:
        支遁,字道林,
        这里尊称为林公。
}%
:「
    澄以石虎为海鸥鸟%
    \footnote{%
        海鸥鸟:
            据《列子,黄帝篇》(一说出自《庄子》)说:
            海边有个人喜欢海鸥,
            天天到海上去跟海鸥玩,
            海鸥都和他很亲近。
            一天他父亲要他借着方便,
            捉一只海鸥回来玩。
            第二天他到海上,
            海鸥再也不飞下来了。
            这里是借此说
            佛图澄
            清净无巧诈之心,不分物我。
    }%
    。
」

%% ----------------------------------------------------------------------------
\switchcolumn

% %% Jy
% 佛图澄和石家的人关系很好,
% 林公支道林说:「
%     佛图澄这是把石虎当作海鸥在交往。
% 」

% %% 妖
