
\switchcolumn[0]*[\section{}]

陶公疾笃%
\footnote{%
    陶公:
        陶侃,谥桓公,字士行(一作士衡)。
        历任湘、广、荆州刺史,
        晋成帝时,封长沙郡公,为太尉,赠大司马,
        名望很高。
}%
,
都无献替之言,
朝士以为恨%
\footnote{%
    都:
        全。
    献替:
        对君主劝善规过、建议兴革。
        按:
        根据王隐所著《晋书》中的记载,
        陶侃临终其实有上表献替,
        这里失实。
    朝士:
        朝廷的官吏。
}%
。
仁祖闻之,曰:「
    时无竖刁,
    故不贻陶公话言%
    \footnote{%
        竖刁:
            春秋时齐桓公所宠信的一名宦官。
            管仲病重时,
            齐桓公问管仲,
            竖刁可否代他做宰相,
            管仲认为此人不能用。
            后来果然发动叛乱。
        贻:
            遗留。
        话言:
            善言,这里指遗言。
    }%
    。
」
时贤以为德音。

%% ----------------------------------------------------------------------------
\switchcolumn

% %% Jy

% %% 妖
