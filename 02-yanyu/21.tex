
\switchcolumn[0]*[\section{}]

诸葛靓在吴,
于朝堂大会%
\footnote{%
    诸葛靓(jìng):
        字仲思,
        他父亲诸葛诞反司马氏,
        被司马昭杀害。
        他入吴国,任右将军、大司马。
        吴亡,逃匿不出。
    朝堂:
        皇帝议政的地方。
}%
。
孙皓问%
\footnote{%
    孙皓:
        吴国末代君主。
}%
:「
    卿字仲思,
    为何所思?
」
对曰:「
    在家思孝,
    事君思忠,
    朋友思信,
    如斯而已%
    \footnote{%
        「如斯」句:有的版本作「如斯而已矣」。
    }%
    。
」

%% ----------------------------------------------------------------------------
\switchcolumn

% %% Jy
% 那时候
% 诸葛靓在吴国,
% 有一次正在朝堂上参加议政大会。
% 吴君孙皓问他:「
%     爱卿字仲思,
%     这仲思仲思,
%     都思些什么呢?
% 」
% 诸葛答道:「
%     臣
%     在家里思虑孝道,
%     工作时思虑忠诚,
%     待朋友思虑信义;
%     臣之所思,
%     不过如此罢了。
% 」

% %% 妖
% 诸葛靓在吴国时,
% 一次朝堂大会上。
% 君主孙皓问他:“
%  爱卿你的字是仲思,
%  是为什么而思念呢?”
% 诸葛靓回答说:“
%  在家思的是孝敬父母,
%  在朝堂思的是忠诚,
%  交友思的是诚实,
%  就是这样。”
