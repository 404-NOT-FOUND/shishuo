
\switchcolumn[0]*[\section{}]

元帝始过江%
\footnote{%
    元帝:
        晋元帝司马睿(ruì),字景文,
        原为琅邪王、安东将军。
        在西晋末年的战乱中,
        国都失守,
        晋愍(mǐn)帝被俘。
        他先过江镇守建康(今南京),
        几年后又在此登位称帝。
        建康原是东吴之地,
        江东士族的势力很大,
        所以有寄人国土之感。
}%
,
谓顾骠骑曰:「
    寄人国土,
    心常怀惭%
    \footnote{%
        顾骠(piào)骑:
            顾荣,字彦先,
            吴人,
            吴亡后到洛阳。
            元帝镇守江东时任军司,加散骑常侍。
            死后赠骠骑将军。
            顾荣是江东士族,
            名望很大,
            所以元帝时他说这番话。
    }%
    。
」
荣跪对曰:「
    臣闻王者以天下为家,
    是以耿、亳无定处,
    九鼎迁洛邑,
    愿陛下勿以迁都为念%
    \footnote{%
        耿、亳(bó):
            商代成汤迁国都到毫邑,
            祖乙又迁到耿邑,
            盘庚再迁回毫邑。
            从成汤到盘庚,
            共迁都五次,
            所以说「无定处」。
        九鼎:
            传说夏禹铸九鼎,
            是传国之宝,
            权力的象征。
            周武王定都镐京,
            却把九鼎迁到同的东都洛邑。
        按:
            顾荣死在元帝即位之前,
            这里不当称陛下。
    }%
    。
」

%% ----------------------------------------------------------------------------
\switchcolumn

% %% Jy
% 晋元帝刚刚过江时,
% 对骠骑将军顾荣说:「
%     寄居他乡,
%     常感羞愧。
% 」
% 顾荣听后,
% 跪将下来:「
%     曾听说
%     真的王者
%     四海为家。
%     商朝
%     五次迁都;
%     周武王
%     刚克服商,
%     就把国宝九鼎迁到洛邑。
%     臣还请陛下
%     不须为迁都而烦忧。
% 」

% %% 妖
% 晋元帝刚到江南时,
% 对骠骑将军顾荣说:「
%  寄居在他人的土地上,
%  让我常常感到惭愧呀。」
% 顾荣跪着回答说:「
%  臣听说王者都是以天下为家的,
%  这也就是商代君王迁都至耿邑与毫邑,没有固定地方的原因吧,
%  周武王也曾将九鼎搬到洛邑,
%  大王您不要再因迁都而耿耿于怀了。」
