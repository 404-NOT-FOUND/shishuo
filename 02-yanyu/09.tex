
\switchcolumn[0]*[\section{}]

南郡庞士元
闻
司马德操在颍川,
故二千里候之%
\footnote{%
    庞士元:
        庞统,字士元,
        东汉末襄阳人,
        曾任南郡功曹(能参与一郡的政务),
        年轻时
        曾去拜会司马德操,
        德操很赏识他,
        称他为凤雏。
        后从刘备。
    司马德操:
        司马徽,字德操,号水镜。
        曾向刘备推荐诸葛亮和庞统。
}%
。
至,
遇德操采桑,
士元从车中谓曰:「
    吾闻丈夫处世,
    当带金佩紫,
    焉有曲洪流之量,
    而执丝妇之事%
    \footnote{%
        带金佩紫:带金印佩紫绶带,指做大官。
                  绶(shòu)带,就是丝带,是用来拴金印的。
                  秦汉时,
                  丞相等大官
                  才有金印紫绶。
        洪流之量:比喻才识气度很大。
    }%
    ?
」
德操曰:「
    子且下车。
    子适知邪径之速,
      不虑失道之迷%
      \footnote{%
          邪径:斜径,小路。
      }%
      。
    昔
    伯成耦耕,
    不慕诸侯之荣%
    \footnote{%
        伯成:
            伯成子高。
            据说尧做君主时,
            伯成子高封为诸侯。
            后来禹做了君主,
            伯成认为禹不讲仁德,
            只讲赏罚,
            就辞去诸侯,
            回家种地。
        耦耕:
            古代的一种耕作方法,
            即两人各扶一张犁,并肩而耕。
            后泛指务农。
    }%
    ;
    原宪桑枢,
    不易有官之宅%
    \footnote{%
        原宪:
            孔子弟子,字子思。
            据说
            他在鲁国的时候,
            很穷,
            住房破破烂烂,用桑树枝做门上的转轴。
            他不求舒适,
            照样弹琴唱歌。
    }%
    。
    何有坐则华屋,
        行则肥马,
        侍女数十,
    然后为奇?
    此乃许、父所以慷慨,
        夷、齐所以长叹%
        \footnote{%
            许、父:
                许由、巢父。
                巢父,
                是许由的朋友,
                尧也想把职位让给他,
                他不肯接受。
            夷、齐:
                伯夷、叔齐,
                商代孤竹君的两个儿子。
                孤竹君死,
                兄弟俩互相让位,不肯继承,
                结果都逃走了。
                后来周武王统一天下,
                两人因反对周武王讨伐商纣,
                不肯吃周朝的粮食,
                饿死在首阳山。
        }%
        。
    虽有窃秦之爵,
    千驷之富,
    不足贵\mbox{也%
    \footnote{%
        窃秦:
            战国末年,
            吕不韦把一个怀孕的妾
            献给秦王子楚,
            生秦始皇赢政。
            赢政登位后,
            尊吕不韦为相国,号称仲父,
            这就是所谓窃秦。
        千驷之富:
            古时候
            用四匹马驾一辆车,
            同拉一辆车的四匹马叫驷。
            千驷,指有一千辆车,四千匹马。
            《论语·季氏》说,
            齐景公有四千匹马,
            可是死了以后,
            人们觉得他没有什么德行值得称赞。
    }}%
    。
」
士元曰:「
    仆
    生出边垂,
    寡见大义,
    若不一叩洪钟,
    伐雷鼓,
    则不识其音响\mbox{也%
    \footnote{%
        边垂:
            即边陲,边疆。
            余嘉锡《世说新语笺疏》中评论说
            庞统来自襄阳(在今湖北襄阳内),
            在汉朝并不算边疆地区;
            又提出
            庞统为人体面,
            不可能见了长辈
            不下车拜见,
            反而在车上说话、
            以及
            由于
            司马徽和庞统的父亲是至交,
            庞统对司马徽的为人也应该有所了解,
            庞统毫无理由
            故意用名利之言去玷污司马、
            从而受教
            等等原由,
            由此怀疑这篇故事是后人编排、并非事实。
        伐:
            敲打。
        雷鼓:
            鼓名,
            古时祭天神时所用的鼓。
    }}%
    !
」

%% ----------------------------------------------------------------------------
\switchcolumn

% %% Jy

% %% 妖

