
\switchcolumn[0]*[\section{}]

过江诸人,
每至美日,
辄相邀新亭,
藉卉饮宴%
\footnote{%
    美日:
        风和日丽的日子。
    新亭:
        也叫劳劳亭,
        原是三国时吴国所筑,故址在今南京市南。
    藉卉(huì):
        坐在草地上。
}%
。
周侯%
\footnote{%
    周侯:即周顗。
}%
中坐
而叹曰:「
    风景不殊,
    正自有山河之异!
」
皆相视流泪。
唯王丞相愀然变色
曰%
\footnote{%
    王丞相:
        王导,字茂弘,
        晋元帝即位后任丞相。
    愀(qiǎo)然:
        形容脸色变得不愉快。
}%
:「
    当共戮力王室,
    克复神州,
    何至作楚囚相对%
    \footnote{%
        戮力:
            并力;合力。
        神州:
            中国,这里指沦陷的中原地区。
        楚囚:
            楚国的囚犯。
            据《左传·成公九年》载:
            一个楚囚弹琴时奏南方乐调,
            表示不忘故旧。
            后来借指处境窘迫的人。
    }%
    !
」

%% ----------------------------------------------------------------------------
\switchcolumn

% %% Jy
% 过江的名士们
% 每逢好天气
% 就要相聚在新亭
% 野餐。
% 周侯周顗在席间
% 突然叹了一口气,
% 说道:「
%     这风景还是一样的风景,
%     可惜江山已经不是一样的江山了啊。
% 」
% 大家听了
% 悲伤难耐,
% 相视流泪。
% 独有王导丞相脸色一沉,
% 怒道:「
%     这正是我们
%     众志成城,
%     保国卫家,
%     夺回主权的时候!
%     你们怎么倒在这里
%     楚楚可怜,
%     像败下阵来的楚囚一样绵软呢!
% 」

% %% 妖
% 从北方逃难到江南的士人们,
% 每逢天气晴朗的好日子,
% 都会相聚在新亭,
% 坐在草地上饮酒作乐。
% 武城侯周顗在席上感叹:“
%  这里的风景与中原无二,
%  只可惜我们的山河易主了!”
% 在座诸人对视数眼,泪流满面。
% 只有丞相王导变了脸色:“
%  我们应该齐心辅佐王世,
%  光复故土,
%  为什么要作出楚人被囚禁在晋国的神态呢!”
