
\switchcolumn[0]*[\section{}]

乐令女适大将军成都王颍%
\footnote{%
    适:
        指女子出嫁。
    成都王颖:
        司马颖,字叔度,
        晋武帝第十六子,
        封成都王,
        后进位大将军。
        在八王之乱中,
        武帝第六子长沙王司马乂(yì)(字士度)
        于公元 301 年入京都,
        拜抚军大将军。
        公元 303 年 8 月,
        司马颖等以司马乂专权,起兵讨伐。
        这里所述就是这一时期内的事。
}%
,
王兄长沙王
执权于洛,
遂构兵相图%
\footnote{%
    构兵:出兵交战。
}%
。
长沙王
亲近小人,远外君子,
凡在朝者,人怀危惧。
乐令既允朝望,加有婚\mbox{亲%
\footnote{%
    允:一说为「处」。
    朝望:在朝廷中有声望。
}}%
,
群小谗于长沙。
长沙尝问乐令,
乐令神色自若,
徐答曰:「
    岂以五男易一\mbox{女%
    \footnote{%
        五男:
            指乐广的五个儿子。
    }}%
    。
」
由是释然,
无复疑虑%
\footnote{%
    按:
        《晋书》载
        长沙王并未信服,
        并解释
        乐广之后不久的离世
        是为此忧虑所致。
}%。

%% ----------------------------------------------------------------------------
\switchcolumn

% %% Jy

% %% 妖

