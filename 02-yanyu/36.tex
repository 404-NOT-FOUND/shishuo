
\switchcolumn[0]*[\section{}]

温峤初为刘琨使
来过江。
于时,
江左营建始尔,
纲纪未举%
\footnote{%
    始尔:始,「尔」是词缀,无实义。
    纲纪:国家的法制。
}%
。
温新至,
深有诸虑。
既诣王丞相,
陈
主上幽越、
社稷焚灭、
山陵夷毁之酷,
有黍离之痛%
\footnote{%
    主上:
        皇帝,
        指晋愍帝司马邺。
        公元 316 年 11 月
        刘曜围长安,
        晋愍帝投降并被赶到平阳。
        317 年 12 月,
        愍帝被杀。
    幽越:
        流亡监禁。
    社稷:
        古代帝王、诸侯所祭的土神和谷神。
        后也借用来泛指国家。
    山陵:
        皇帝的坟墓。
    黍离:
        《诗经·王风》篇名,
        据说周王室迁到东都洛阳以后,
        有人到西部,
        看到原来的宗庙宫室已经毁为平地,
        种上了黍稷,
        哀怜周王室日渐衰微,
        心里忧伤,
        便作了这首诗。
}%
。
温
忠慨深烈,
言与泗俱%
\footnote{%
    泗(sì):鼻涕。
}%
;
丞相亦与之对泣。
叙情既毕,
便深自陈结,
丞相亦厚相酬纳。
既出,
欢然言曰:「
    江左自有管夷吾%
    \footnote{%
        管夷吾:
            字仲,
            春秋时代齐国人,
            齐桓公的宰相,
            辅助齐桓公成为霸王。
    }%
    ,
    此复何忧!
」

%% ----------------------------------------------------------------------------
\switchcolumn

% %% Jy

% %% 妖

