
\newpage
\switchcolumn[0]*[\section{}]

孔文举
年十岁,随父到洛%
\footnote{%
    孔文举:孔融,
            字文举,
            是汉代末年的名士、文学家,
            曾多次反对曹操,
            被曹操借故杀害。
}%
。
时李元礼有盛名,
为司隶校尉%
\footnote{%
    司隶校尉:官名,掌管监察京师和所瞩各郡百官的职权。
}%
。
诣门者,皆
    俊才清称及
    中表亲戚
乃通%
\footnote{%
    清称:有清高的称誉的人。
}%
。
文举至门,
谓吏曰:「
    我是李府君亲。
」
既通,前坐。
元礼问曰:「
    君与仆有何亲%
    \footnote{%
        仆:谦称。
    }%
    ?
」
对曰:「
    昔
    先君仲尼与君先人伯阳
    有师资之尊,
    是
    仆与君奕世为通好也%
    \footnote{%
        先君:祖先,与下文「先人」同。
        仲尼:孔子,名丘,字仲尼。
        伯阳:老子,姓李,名耳,字伯阳。
        师资:师。这里指孔子曾向老子请教过礼制的事。
        奕世:累世;世世代代。
    }%
    。
」
元礼及宾客莫不奇之。
太中大夫陈韪后至,
人以其语语之%
\footnote{%
    太中大夫:掌管议论的官。
    陈韪(wěi):《后汉书·孔融传》作陈炜,字不详。
}%
,
韪曰:「
    小时了了,
    大未必佳%
    \footnote{%
        了了:聪明;明白通晓。
    }%
    !
」
文举曰:「
    想君小时,
    必当了了!
」
韪大踧踖%
\footnote{%
    踧踖(cùjí):局促不安的样子。
}%
。

%% ----------------------------------------------------------------------------
\switchcolumn

% %% Jy

% %% 妖

