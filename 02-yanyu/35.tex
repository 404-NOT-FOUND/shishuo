
\switchcolumn[0]*[\section{}]

刘琨虽隔阂寇戎,
志存本朝%
\footnote{%
    刘琨:
        字越石,
        封广武侯,
        西晋未年,
        拜司空,
        出任并州刺吏,
        都督并、冀、幽三州军事,
        有志辅佐帝室,平定北方。
        公元 316 年,京都失陷,
        317 年司马睿在江南称晋王,
        这时刘琨仍在北方,
        便派下属温峤到建康上表劝进。
    寇戎:
        入侵的外族。
        戎,我国西部少数民族。
        西晋未诸王侯争权,互相攻伐,
        北部和西部各族也乘机侵入中原。
}%
。
谓温峤曰%
\footnote{%
    温峤(qiáo):
        字太真,
        在刘琨手下
        任右司马
        (军府的官职,综理一府之事)。
}%
:「
    班彪识刘氏之复兴,
    马援知汉光之可辅%
    \footnote{%
        班彪:
            字叔皮,
            汉代人。
            开始时追随隗嚣,
            隗嚣想叛离汉光武帝刘秀,
            班彪曾反对。
            后追随窦融,
            融初依附淮阳王,
            班彪为他谋划归附光武帝。
            西汉末
            王莽篡位,
            改国号为新。
            后来刘秀即位,
            定都洛阳,
            汉室复兴。
        马援:
            字文渊,
            汉代人。
            见了光武帝以后,
            说他是真帝王。
            封新息侯,
            拜伏波将军,
            辅佐汉光武帝,
            南征北伐,
            屡建战功。
    }%
    。
    今
    晋祚虽衰,
    天命未改%
    \footnote{%
        晋阼:晋王朝的国统。
    }%
    ,
    吾欲立功于河北,
    使卿延誉于江南,
    子其行乎%
    \footnote{%
        延:传播。
    }%
    ?
」
温曰:「
    峤虽不敏,
    才非昔人,
    明公以桓、文之姿,
    建匡立之功,
    岂敢辞命!
」

%% ----------------------------------------------------------------------------
\switchcolumn

% %% Jy

% %% 妖

