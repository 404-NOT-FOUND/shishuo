
\switchcolumn[0]*[\section{}]

周仆射雍容好仪形%
\footnote{%
    周仆射:指周顗。
    雍容:形容举止大方,温和从容。
}%
。
诣王公,
初下车,
隐数人,
王公含笑看之%
\footnote{%
    隐(yìn):
        依靠。
        按:
        当时出入要人搀扶,
        这是贵族的习惯。
}%
。
既坐,
傲然啸咏%
\footnote{%
    傲然:
        形容傲慢没礼貌。
    啸咏:
        啸是吹口哨,咏是歌咏,
        即吹出曲调。
        啸咏是当时文士一种习俗,
        更是放诞不羁、傲世的人
        表现其名士风流的一种姿态。
}%
。
王公曰:「
    卿欲希嵇、阮邪%
    \footnote{%
        希:企望;仰慕。
        嵇、阮:嵇康、阮籍。
    }%
    ?
」
答曰:「
    何敢
    近舍明公,
    远希嵇、阮!
」

%% ----------------------------------------------------------------------------
\switchcolumn

% %% Jy
% 仆射周顗举止大方、仪表堂堂。
% 他去诣访王导丞相,
% 一下车就有好几个人来搀扶他,
% 王丞相含笑看着他。
% 坐下以后,
% 周顗旁若无人地吹起了口哨,
% 王公问他:「
%     你这是在学嵇康、阮籍吗?
% 」
% 周仆射答道:「
%     我面前就有明公可以效仿,
%     哪里还要去学什么嵇康、阮籍!
% 」

% %% 妖
