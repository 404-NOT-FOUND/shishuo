
\switchcolumn[0]*[\section{}]

颍川太守髡陈仲弓%
\footnote{%
    髡(kūn):古代一种剃去男子头发的刑罚。
    按:陈寔被捕两次,
        一次受冤入狱,
        这一次是在任太丘后,
        他涉及党锢之祸
        自请入狱,
        并说
        「吾不就狱,众无所恃」;
        后遇赦放出。
}%
。
客有问元方:「府君如何?」
元方曰:「高明之君也。」
「足下家君如何?」
曰:「忠臣孝子也。」
客曰:「
    易称:『
        二人同心,其利断金;
        同心之言,其臭如兰。
    』
    何有高明之君,
    而刑忠臣孝子者乎?
」
元方曰:「
    足下言何其谬也!
    故不相答。
」
客曰:「
    足下但因伛为恭而不能答%
    \footnote{%
        伛(yǔ):驼背弯腰。
    }%
    。
」
元方曰:「
    昔高宗放孝子孝己,
    尹吉甫放孝子伯奇,
    董仲舒放孝子符起%
    \footnote{%
        孝己:殷代君主高宗武丁的儿子,
              他侍奉父母最孝顺,
              后来高宗受后妻的迷惑,
              把孝己放逐致死。
        伯奇:周代的卿士(王朝执政官)尹吉甫的儿子,
              对后母非常孝顺,
              却受到后母诬陷,
              被父亲放逐。
              伯奇的下场说法不一;
              曹植说伯奇为父所杀,
              又有人说吉甫后来醒悟,寻回伯奇并射杀了后妻。
        符起:其事不详。
    }%
    。
    唯此三君,高明之君;
    唯此三子,忠臣孝子。
」
客惭
而退。

%% ----------------------------------------------------------------------------
\switchcolumn

% %% Jy

% %% 妖

