
\switchcolumn[0]*[\section{}]

庾公造周伯仁%
\footnote{%
    庚公:
        庚亮,字元规,
        晋成帝之舅,
        成帝朝辅政,
        任给事中,徙中书令。
    周伯仁:
        周顗(yǐ),字伯仁,
        袭父爵武城侯,世称周侯,
        曾任吏部尚书,尚书左仆射。
        有才华有正气,
        汝南贲泰渊曾说:「
            伯仁将袪旧风,清我邦族矣。
        」
}%
,
伯仁曰:「
    君何所欣说而忽肥?
」
庾曰:「
    君复何所忧惨而忽瘦?
」
伯仁曰:「
    吾无所忧,
    直是清虚日来,
    滓秽日去耳%
    \footnote{%
        直是:只是。
        清虚:清静淡泊。
        滓秽:污秽;丑恶。
    }%
    。
」

%% ----------------------------------------------------------------------------
\switchcolumn

% %% Jy
% 庚亮庚公去造访周伯仁,
% 伯仁问:「
%     您遇见什么好事儿啦?
%     一会儿不见,胖了一圈儿呢!
% 」
% 庚公说:「
%     那您遇见什么坏事儿啦?
%     一会儿不见,瘦了一圈儿呢!
% 」
% 周伯仁回答道:「
%     并不是什么不好的事情;
%     只不过是
%     渐渐地
%     学会了清静淡泊,
%     远离了污秽罢了。
% 」

% %% 妖
% 庾亮去拜访周伯仁,
% 周伯仁说:“
%  您为何而欢喜,怎么一下圆润了起来呢?”
% 庾亮说:“
%  那您又是因何忧虑,突然消瘦了呢?”
% 周伯仁回答:“
%  我并无烦心事,
%  只是一天天变得清静淡泊起来,
%  那些污浊的思虑与日俱减罢了。”
