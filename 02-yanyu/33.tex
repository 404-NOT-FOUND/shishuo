
\switchcolumn[0]*[\section{}]

顾司空未知名,
诣王丞相%
\footnote{%
    顾司空:
        顾和,字君孝。
        少年成才,
        顾荣曾说他「
            吾家之骐骥也,
            必振衰族
        」。
        王导任扬州刺史时,
        召他为从事,
        累迁尚书令,
        死后追赠司空。
}%
。
丞相小极,
对之疲睡%
\footnote{%
    极:疲乏。
    疲睡:打瞌睡。
}%
。
顾思所以叩会之,
因谓同坐曰:「
    昔
    每闻元公道公协赞中宗,
    保全江表%
    \footnote{%
        元公:
            指顾荣,
            他是顾和的族叔。
            顾荣死后,
            溢号为元,
            所以称为元公。
        协赞中宗:
            中宗是
            晋元帝的庙号。
            王导和元帝关系很好,
            知道乱世将至,
            劝元帝渡江,回到自己的封国。
            邓粲在《晋纪》中说
            「
                晋中兴之功,
                导实居其首
            」。
            按:
            顾和初出仕是在元帝时,
            还不可能有元帝的庙号。
        江表:
            长江之外,即江南。
    }%
    。
    体小不安,
    令人喘息。
」
丞相因觉,
谓顾曰:「
    此子珪璋特达,
    机警有锋%
    \footnote{%
        珪璋(guī zhāng)特达:
            珪和璋是玉器,
            是诸侯朝见天子时所用的重礼。
            用珪璋时可以单独送达,
            不须加上别的礼品为辅。
            后用来比喻
            有才德的人
            不用别人推荐也会有成就。
    }%
    。
」

%% ----------------------------------------------------------------------------
\switchcolumn

% %% Jy
% 司空顾和还不知名的时候,
% 去造访王丞相。
% 丞相
% 当时正疲惫,
% 当着他的面就打起了瞌睡。
% 顾和
% 想要引起王丞相的注意,
% 于是便对同坐的人说:「
%     以前常听
%     元公顾荣
%     说起王公
%     辅佐中宗,
%     保全江南
%     的故事。
%     现在看到王公身体不太舒适,
%     真是让人喘息不安。
% 」
% 丞相听了
% 一下醒过来,
% 说:「
%     这孩子
%     珪璋特达,
%     机警犀利。
% 」

% % %% 妖
% 当顾和还是无名小卒时,
% 他去拜访了丞相王导。
% 王导有些疲惫,
% 就对着他打瞌睡。
% 顾和思考该如何不留痕迹地叫醒王导并打招呼
% 就对着同坐诸位说:“
%  当时每每听闻元公称赞丞相辅佐中宗
%  保全了江南。
%  现如今丞相贵体欠安,
%  真令人担忧啊。”
% 王导因此醒来,
% 对顾和说:“
%  你定非池中之物,
%  反应敏捷,言辞有力。”
