
\switchcolumn[0]*[\section{}]

荀慈明与汝南袁阆
相见,
问颍川人士,
慈明先及诸兄。
阆笑曰:「
    士但可因亲旧而已乎?
」
慈明曰:「
    足下相难,
    依据者何经%
    \footnote{%
        经:常规;原则。或作「依据者何因」。
    }%
    ?
」
阆曰:「
    方问国士,
    而及诸兄,
    是以尤之耳%
    \footnote{%
        国士:全国推崇的才德之士。
        尤:指责;责问。
    }%
    。
」
慈明曰:「
    昔者祁奚
    内举不失其子,
    外举不失其雠,
    以为至公%
    \footnote{%
        祁奚:
            春秋时代晋国人,
            任中军尉(掌管军政的长官)。
            祁奚告老退休,
            晋悼公问他接班人的人选,
            他推荐了他的仇人解狐。
            刚要任命,解狐却死了。
            晋悼公又问祁奚,
            祁奚推荐自己的儿子祁午。
            大家称赞祁奚能推荐有才德的人。
        雠(chóu):同「仇」,指仇人。
    }%
    。
    公旦《文王》之诗,
    不论尧、舜之德而颂文、武者,
    亲亲之义也%
    \footnote{%
        公旦:
            周公旦。
            周公,姓姬,名旦,
            是周文王第四子,周武王的弟弟,周成王的叔父,
            辅助周成王。
        《文王》:
            指《诗经·大雅·文王之什》,
            包括《文王》、《大明》等十篇,
            分别歌颂文王、武王之德。
            作者无考,
            《文王》一篇,有以为周公所作。
        亲亲:爱亲人。
    }%
    。
    《春秋》之义,
    内其国而外诸夏%
    \footnote{%
        《春秋》:
            儒家经典之一。
            是春秋时代鲁国的史书,
            也是我国第一本编年体史书。
        诸夏:古时指属于汉民族的各诸侯国。
    }%
    。
    且不爱其亲而爱他人者,
    不为悖德乎%
    \footnote{%
        悖(bèi):违背。
    }%
    ?
」

%% ----------------------------------------------------------------------------
\switchcolumn

%% Jy

% %% 妖

