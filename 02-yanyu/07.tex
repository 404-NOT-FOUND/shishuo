
\switchcolumn[0]*[\section{}]

荀慈明与汝南袁阆
相见,
问颍川人士,
慈明先及诸兄。
阆笑曰:「
    士但可因亲旧而已乎?
」
慈明曰:「
    足下相难,
    依据者何经%
    \footnote{%
        经:常规;原则。或作「依据者何因」。
    }%
    ?
」
阆曰:「
    方问国士,
    而及诸兄,
    是以尤之耳%
    \footnote{%
        国士:全国推崇的才德之士。
        尤:指责;责问。
    }%
    。
」
慈明曰:「
    昔者祁奚
    内举不失其子,
    外举不失其雠,
    以为至公%
    \footnote{%
        祁奚:
            春秋时代晋国人,
            任中军尉(掌管军政的长官)。
            祁奚告老退休,
            晋悼公问他接班人的人选,
            他推荐了他的仇人解狐。
            刚要任命,解狐却死了。
            晋悼公又问祁奚,
            祁奚推荐自己的儿子祁午。
            大家称赞祁奚能推荐有才德的人。
        雠(chóu):同「仇」,指仇人。
    }%
    。
    公旦《文王》之诗,
    不论尧、舜之德而颂文、武者,
    亲亲之义也%
    \footnote{%
        公旦:
            周公旦。
            周公,姓姬,名旦,
            是周文王第四子,周武王的弟弟,周成王的叔父,
            辅助周成王。
        《文王》:
            指《诗经·大雅·文王之什》,
            包括《文王》、《大明》等十篇,
            分别歌颂文王、武王之德。
            作者无考,
            《文王》一篇,有以为周公所作。
        亲亲:爱亲人。
    }%
    。
    《春秋》之义,
    内其国而外诸夏%
    \footnote{%
        《春秋》:
            儒家经典之一。
            是春秋时代鲁国的史书,
            也是我国第一本编年体史书。
        诸夏:古时指属于汉民族的各诸侯国。
    }%
    。
    且不爱其亲而爱他人者,
    不为悖德乎%
    \footnote{%
        悖(bèi):违背。
    }%
    ?
」

%% ----------------------------------------------------------------------------
\switchcolumn

%% Jy
% 荀慈明与汝南的袁奉高相见,
% 谈及颖川名士,
% 慈明马上提到他的几个哥哥。
% 袁奉高大笑,说:「
%     所谓名士
%     莫非都只是靠
%     亲戚朋友相互吹捧
%     而已吗?
% 」
% 慈明说:「
%     我不知道先生何故发难?
% 」
% 袁奉高解释:「
%     我刚一问起贤人名士,
%     你马上就说起自己的兄弟们。
%     我对这样的做法
%     不很看好。
% 」
% 慈明听后说:「
%     从前有个人叫祁奚,
%     推士举贤的时候,
%     从亲生儿子、
%     到他的死对头
%     他都不偏不倚地推荐过。
%     人们都说他公平之极。
%     周公旦作下了《文王》一诗,
%     没有歌颂尧舜的大德,
%     而歌颂的是自己的父亲,
%     这也是符合大义、
%     体现了他们对亲人的爱。
%     就连《春秋》
%     也把本国  的内容看成是内事、
%       把诸侯国的内容看成是外务。
%     再者说了,
%     不爱自己的亲人
%     反而
%     先去亲和外人,
%     这样的人
%     难道不是悖德之人吗?
% 」

% %% 妖
% 荀慈明与汝南郡的袁阆相见时,
% 袁阆问及颍川的贤达人士,
% 荀慈明先说起了他的几位兄长。
% 袁阆就笑着说:“
%  贤德之士难道是以亲疏远近来判断的嘛?”
% 荀慈明说:“
%  您这样责难我,
%  是有什么依据嘛?”
% 袁阆说:“
%  我刚才问的是国之重士,
%  而你回答的确是你的兄长,
%  所以我才责问你。”
% 慈明回答说:“
%  从前的祁奚,
%  举荐人才对内不避讳他的儿子,
%  对外也不忌讳他的仇人。
%  周公旦做《文王》时,
%  也不去论述尧舜的功德反而歌颂周文王、武王,
%  这是符合爱亲人这一大义的。
%  《春秋》记事的原则,
%  就是把本国的事情看成亲近的,
%  而将其他诸侯国的事看做疏远的。
%  何况不爱自己的亲人却去爱其他人的人,
%  难道不是有违道德呢?
 
