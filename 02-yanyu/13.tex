
\switchcolumn[0]*[\section{}]

魏明帝为外祖母筑馆于甄氏%
\footnote{%
    魏明帝:
        即曹睿(ruì),字元仲,
        文帝曹丕的儿子。
        初文帝废后,
        曹睿并不是太子;
        一次他们同去打猎,
        见到子母双鹿,
        文帝射死母鹿,
        曹睿则不忍再射小鹿。
        文帝认为他很善良,
        于是定为太子。
    馆:华丽的房屋。
    甄氏:明帝的母亲姓甄,这里指甄家。
}%
,
既成,自行视,
谓左右曰:
「馆当以何为名?」
侍中缪袭曰:「
    陛下圣思
    齐于哲王,
    罔极过于曾、闵。
    此馆之兴,
    情钟舅氏,
    宜以『渭阳』为名%
    \footnote{%
        缪袭:
            字熙伯,
            有才学,
            官至光禄勋。
        圣思:皇帝的思虑。
        哲王:贤明的君主。
        罔极:
            无极;无穷无尽。
            这里用《诗经·小雅·蓼莪》
            「欲报之德,吴天罔极」之意,
            指父母的恩德象天那样无穷无尽,难以报答。
        曾、闵(mǐn):
            曾指曾子,名参(shēn);
            闵指闵子骞。
            二人都是孔子的学生,是古时著名的孝子。
        钟:集中。
        渭阳:
            渭水北边。
            语出《诗经·秦风·渭阳》:
            「我送舅氏,曰至渭阳」
            (我送舅舅,送到渭水北边)。
            这首诗据说是春秋时
            秦康公为送别舅舅(晋文公重耳)
            而思念亡母时作的,
            后人以此说明舅甥之情。
            明帝之母甄氏被文帝曹丕赐死,
            明帝为舅家建馆,
            也是为纪念亡母,
            因此缪袭以为应该根据这两句诗的意思来起名。
        按:
            《魏书》记载,
            魏明帝给舅母修了一所楼馆,
            并不是给外祖母修的。
    }%
    。
」

%% ----------------------------------------------------------------------------
\switchcolumn

% %% Jy

% %% 妖

