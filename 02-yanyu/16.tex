
\switchcolumn[0]*[\section{}]

司马景王东征,
取上党李喜,
以为从事中郎%
\footnote{%
    司马景王:
        司马师,字子元,
        三国时魏人,
        司马懿的儿子,
        封长平乡侯,曾任大将军,
        辅助齐王曹芳,后又废曹芳,立曹髦(máo)。
        毌(guàn)丘俭起兵反对他,被他打败。
        这里说的东征,就是指的这件事。
        晋国建立,追尊为景王。
        后来晋武帝司马炎上尊号为景帝。
    李喜:
        字季和,
        上党郡人。
        司马懿任相国时,
        召他出来任职,
        他托病推辞。
        司马师时代官拜特进、光禄大夫,
        死后追赠太保。
    从事中郎:
        官名,
        大将军府的属官,参与谋议等事。
}%
。
因问喜曰:「
    昔
    先公辟君不就,
    今孤召君,
    何以来%
    \footnote{%
        先公:称自己或他人的亡父。
        辟:征召。
        就:到。
        孤:侯王的谦称。
    }%
    ?
」
喜对曰:「
    先公以礼见待,
    故得以礼进退;
    明公以法见绳,
    喜畏法而至耳%
    \footnote{%
        明公:对尊贵者的敬称。
    }%
    。
」

%% ----------------------------------------------------------------------------
\switchcolumn

% %% Jy
% 景王司马师东征伐毌丘,
% 招募了上党郡人李季和,
% 拜他为从事中郎。
% 他问李季和:「
%     旧时家父先公请您来做官,
%     您推托不来。
%     怎么今天我来找您,
%     您就来了呢?
% 」
% 李季和答道:「
%     先公他以礼相待,
%     我由此得以自由决定;
%     今天明公您用法令来召我,
%     我受到法律限制,不敢不来呀。
% 」

% %% 妖

