
\switchcolumn[0]*[\section{}]

王敦兄含,
为光禄勋%
\footnote{%
    王敦:
        晋室东迁,
        与堂兄弟王导一起辅佐晋元帝,
        任大将军、荆州刺史,
        镇守武昌。
        公元 322 年以武昌起兵谋反,入建康。
        当时晋元帝命王导为前锋大部督
        抵抗王敦。
        后元帝任王敦为丞相,
        他伪辞不受,
        始返武昌。
    含:
        王含,字处弘,
        累迁至光䘵勋,
        与王敦一同谋反,
        被诛。
    光禄勋:
        官名,
        掌管皇帝宿卫侍从。
}%
。
敦既逆谋,
屯据南州,
含委职奔姑孰%
\footnote{%
    逆谋:
        疑为「谋逆」误写。
    委:
        推卸、抛弃。
    姑孰:
        在今安徽省当涂县。
        
}%
。
王丞相诣阙谢%
\footnote{%
    阙:京城、宫殿。
    谢:认错。
}%
。
司徒、丞相、扬州官僚问讯,
仓卒不知何辞%
\footnote{%
    官僚:官府所统属的官吏。
          按:
          王导时为扬州刺史兼司空,
          还不是司徒、丞相。
          这里不合实情。
    仓卒:仓猝。
}%
。
顾司空时为扬州别驾,
援翰曰%
\footnote{%
    别驾:
        官名,
        刺史的属官,
        职务是随刺史外出视察。
    翰:
        笔。
}%
:「
    王光禄
    远避流言,
    明公
    蒙尘路次,
    群下不宁,
    不审尊体起居何如%
    \footnote{%
        蒙尘:
            蒙受风尘。
            指王导天天诣阙谢罪。
        路次:
            路中。
        群下:
            僚属;部下。
        起居:
            日常生活。
    }%
    ?
」

%% ----------------------------------------------------------------------------
\switchcolumn

% %% Jy
% 王敦谋图造反,
% 领兵驻扎在南州。
% 他的哥哥光禄勋王含
% 知道以后
% 就弃官投奔到姑熟去了。
% 丞相王导
% 与王敦兄弟也有亲戚关系,
% 为此就到王宫朝廷上去请罪。
% 司徒和丞相门下以及扬州府的官员们
% 都来打探消息,
% 着急忙慌得不知该如何措辞。
% 那时候,
% 司空顾和还在扬州做别驾,
% 提笔写来:「
%     王含王光禄
%     早早地就避开了流言蜚语,
%     明公您
%     却要天天风尘仆仆地上朝请罪,
%     下属们心神难宁,坐立不安。
%     不知您的饮食起居还好不好?
% 」

% %% 妖
% 王敦的哥哥王含,
% 担任了光禄勋一职。
% 当王敦谋反以后,
% 领兵驻扎在南州,
% 王含就弃官直奔姑熟而去。
% 丞相王导为此带着全家在朝廷上请罪。
% 司徒和丞相门下以及扬州府的官员都来打探消息,
% 着急忙慌的不知该如何措辞。
% 此时,司空顾和作为扬州刺史的属官,
% 拿起笔来写到:“
%  光禄大人您远远地避开了留言,
%  如今丞相天天风尘仆仆地请罪,
%  下属们也坐立不安,
%  不知您的饮食起居还好嘛?”
