
\switchcolumn[0]*[\section{}]

王敦兄含,
为光禄勋%
\footnote{%
    王敦:
        晋室东迁,
        与堂兄弟王导一起辅佐晋元帝,
        任大将军、荆州刺史,
        镇守武昌。
        公元 322 年以武昌起兵谋反,入建康。
        当时晋元帝命王导为前锋大部督
        抵抗王敦。
        后元帝任王敦为丞相,
        他伪辞不受,
        始返武昌。
    含:
        王含,字处弘,
        累迁至光䘵勋,
        与王敦一同谋反,
        被诛。
    光禄勋:
        官名,
        掌管皇帝宿卫侍从。
}%
。
敦既逆谋,
屯据南州,
含委职奔姑孰%
\footnote{%
    逆谋:
        疑为「谋逆」误写。
    委:
        推卸、抛弃。
    姑孰:
        在今安徽省当涂县。
        
}%
。
王丞相诣阙谢%
\footnote{%
    阙:京城、宫殿。
    谢:认错。
}%
。
司徒、丞相、扬州官僚问讯,
仓卒不知何辞%
\footnote{%
    官僚:官府所统属的官吏。
    仓卒:仓猝。
}%
。
顾司空时为扬州别驾,
援翰曰%
\footnote{%
    别驾:
        官名,
        刺史的属官,
        职务是随刺史外出视察。
    翰:
        笔。
}%
:「
    王光禄
    远避流言,
    明公
    蒙尘路次,
    群下不宁,
    不审尊体起居何如%
    \footnote{%
        蒙尘:
            蒙受风尘。
            指王导天天诣阙谢罪。
        路次:
            路中。
        群下:
            僚属;部下。
        起居:
            日常生活。
    }%
    ?
」

%% ----------------------------------------------------------------------------
\switchcolumn

% %% Jy

% %% 妖

