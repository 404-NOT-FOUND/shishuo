
\switchcolumn[0]*[\section{}]

谢仁祖
年八岁,
谢豫章
将送客%
\footnote{%
    谢仁祖:
        谢尚,字仁祖,
        谢鲲的儿子,
        后任镇西将军、豫州刺史。
    谢豫章:
        谢鲲,
        曾任豫章太守。
    将:
        带领。
}%
。
尔时
语已神悟,
自参上流%
\footnote{%
    自参上流:
        自处于上等名流之中。
}%
。
诸人咸共叹之,曰:「
    年少,
    一坐之颜回%
    \footnote{%
        颜回:
            春秋时鲁国人,
            对孔子的学说深有体会,
            孔子很赏识他。
    }%
    。
」
仁祖曰:「
    坐无尼父%
    \footnote{%
        尼父(fǔ):
            孔子。
            因孔子字仲尼,故尊称为尼父,也作「尼甫」。
    }%
    ,
    焉别颜回?
」

%% ----------------------------------------------------------------------------
\switchcolumn

% %% Jy
% 谢仁祖八岁的时候,
% 他的父亲豫章太守谢鲲带着他送客。
% 当时他的言语已经透出不凡的悟性,
% 自居于上流人士。
% 大家都感叹说
% 他年纪轻轻,
% 已经是在座的一个小颜回了。
% 他却不认同:「
%     座中又没有孔子,
%     你怎么知道哪个是颜回?
% 」

% %% 妖
