
\switchcolumn[0]*[\section{}]

王武子、孙子荆
各言其土地人物之美%
\footnote{%
    王武子:
        王济,字武子,太原晋阳人,
        有俊才,能清言,
        官至太仆。
    孙子荆:
        孙楚,字子荆,太原中都人,
        仕至冯诩太守。
}%
。
王云:「
    其地坦而平,
    其水淡而清,
    其人廉且贞。
」
孙云:「
    其山㠑嵬以嵯峨%
    \footnote{%
        㠑(zuì)嵬:
            山险峻的样子。
    }%
    ,
    其水㳌渫而扬波%
    \footnote{%
        㳌渫(jiá dié):
            水波连续的样子。
    }%
    ,
    其人磊砢而英多%
    \footnote{%
        磊砢(luǒ):
            形容魁礨(lěi),即累积众多的样子。
            《文选·上林赋》:「水玉磊砢」。
            水玉,水晶石。
    }%
    。
」

%% ----------------------------------------------------------------------------
\switchcolumn

% %% Jy
% 王武子和孙子荆
% 正在聊自己家乡的风景和人物。
% 王武子说:「
%     我的家乡晋阳
%     土地平坦而开阔,
%     河水清淡而澄澈,
%     人民廉洁而公正。
% 」
% 孙子荆说:「
%     我的家乡中都
%     山势巍峨而险峻,
%     河水汹涌而浩荡,
%     人才杰出而众多。
% 」

% %% 妖
% 王武子和孙子荆分别谈论其家乡风土人情的出色之处。
% 王武子说:“
%  我的家乡晋阳土地开阔又平坦,
%  河水清淡又澄澈,
%  人民廉洁又公正。”
% 孙子荆说:“
%  我的家乡中都山势巍峨险峻,
%  河水浩荡汹涌,
%  人才辈出。”
