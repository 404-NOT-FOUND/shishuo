
\switchcolumn[0]*[\section{}]

诸名士共至洛水戏,
还,乐令问王夷甫曰%
\footnote{%
    乐令:
        乐广,字彦辅,
        官至任尚书令,故称乐令。
    王夷甫:王衍,字夷甫,曾任太尉。
}%:「
    今日戏
    乐乎?
」
王曰:「
    裴仆射善谈名理,
    混混有雅致%
    \footnote{%
        裴仆射:
            裴頠(wěi),字逸民,
            历任侍中、尚书左仆射。
        名理:
            考核名实,辨别、分析事物是非,
            道理之学,
            是魏晋清谈的主要内容。
        混混:
            滚滚,形容说话滔滔不绝。
    }%;
    张茂先论史、汉,
    靡靡可听%
    \footnote{%
        张茂先:
            张华,字茂先,
            博览群书,
            晋武帝时任中书令,封广武候。
        靡靡:
            娓娓,动听的样子。
    }%;
    我与王安丰说延陵、子房,
    亦超超玄著%
    \footnote{%
        王安丰:
            即王戎,封安丰侯。
        延陵:
            今江苏武进县,这里以地代人。
            春秋时吴王寿梦的少子季礼
            封在这里,
            称为延陵季子。
            有贤名,
            吴王欲立之,辞不受。
        子房:
            张良,字子房;
            本战国时韩国人,
            秦灭韩,张良以全部家产求刺客刺秦王。
            后帮助刘邦击败项羽,封为留侯。
        超超玄著:
            指议论超尘拔俗,奥妙透彻。
    }%。
」

%% ----------------------------------------------------------------------------
\switchcolumn

% %% Jy
% 几位名人雅士们
% 一共到洛水去游玩,
% 回来的路上,
% 尚书令乐广问甫王衍
% 玩得怎么样,
% 王衍说:「
%     裴逸民善于论理,
%     滔滔不绝、
%     致趣高雅;
%     张茂先攀谈汉史,
%     娓娓相述、
%     鲜活动听;
%     我与王安丰讨论延陵季子、韩国张良,
%     也是非常
%     超凡脱俗、
%     奥妙深刻的呀。
% 」

% %% 妖

