
\switchcolumn[0]*[\section{}]

邓艾口吃,
语称艾艾%
\footnote{%
    邓艾:
        字士载,
        曾自名邓范,字士则,
        取「言为世范,行为士则」意,
        后改为邓艾。
        三国时魏人,
        司马懿召为属官,
        后任镇西将军,又封邓侯。
        与钟会一同伐蜀,
        他偷渡阴平,逼刘禅投降,建立灭蜀奇功。
        后遭同僚钟会陷害,
        在钟会之乱中被卫瓘所杀。
        唐后被推崇为古今六十四将之一。
    艾艾:
        古代和别人说话时,多自称名。
        邓艾因为口吃,自称时就会连说「艾艾」。
}%
。
晋文王戏之曰%
\footnote{%
    晋文王:
        司马昭,字子上,
        司马懿次子。
}%
:「
    卿云艾艾,
    定是几艾?
」
对曰:「
    『凤兮凤兮』,
    故是一凤%
    \footnote{%
        凤兮凤兮:
            语出《论语·微子》,
            说是楚国的陆通,即接舆(jiēyú),
            走过孔子身旁的时候
            唱道:
            「凤兮风兮,何德之衰」,
            这里以凤比喻孔子。
    }%
    。
」

%% ----------------------------------------------------------------------------
\switchcolumn

% %% Jy
% 邓艾有口吃的毛病,
% 说话自称的时候,
% 老是把「艾」说成「艾艾」。
% 晋文王一次和他打趣儿,问他
% 「
%     你总说艾艾、艾艾的,
%     那到底是有几个艾啊?
% 」
% 邓艾只说:「
%     古人虽说
%     『凤兮凤兮』,
%     其实呢,
%     说的只有一只凤而已。
% 」

% %% 妖
% 郑艾有口吃之疾,
% 每每自称必言“艾艾”。
% 晋文王因此取笑他:“
% 你说艾艾,
% 到底有几个艾呢?”
% 郑艾回答:“
%  楚国路通经过孔子身旁吟诵“凤兮凤兮”
%  也只有一个孔子呀。”
