
\switchcolumn[0]*[\section{}]

郗太尉拜司空,
语同坐曰%
\footnote{%
    郗太尉:
        郗鉴。
        晋成帝咸和四年(公元 329 年)任司空,
        后又进位太尉。
}%
:「
    平生意不在多,
    值世故纷纭,
    遂至台\mbox{鼎%
    \footnote{%
        台鼎:
            指三公或宰相。
            东汉时
            大尉、司徒、司空
            合称三公,
            是最高的官位。
            人们拿三台(星名)和鼎足
            来比喻三公,
            说成台鼎。
    }}%
    。
    朱博翰音,
    实愧于怀%
    \footnote{%
        朱博:
            字子元,
            汉代人,
            出任丞相,
            临授职时,
            忽然有一种像钟声的声音响起。
            李寻解释说,
            是因为
            君主治理无方,
            有名无实的人得以进位,
            才会有一种无形的东西
            发出声音。
            这里比喻名不副实,
            不应处此高位。
        翰音:
            翰指高飞,
            声音高飞,
            比喻空名。
    }%
    。
」

%% ----------------------------------------------------------------------------
\switchcolumn

% %% Jy
% 郗鉴太尉官拜司空的时候,
% 对周围的人说:「
%     我本没有奢求这么多金钱地位,
%     赶上乱世,
%     才升到
%     这
%     台鼎之列,
%     三公之中。
%     可比
%     汉代朱博,
%     空有虚名,
%     实在惭愧不堪。
% 」

% %% 妖
% 太尉郗鉴任司空的时候,
% 对周围的人说:“
%  我平生的志向不高,
%  只是遭遇乱世,
%  侥幸位居三公。
%  想起朱博徒有虚名,他担任丞相时所遭遇的钟声,
%  只觉心中惭愧。”
