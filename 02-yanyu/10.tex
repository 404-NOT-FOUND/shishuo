
\switchcolumn[0]*[\section{}]

刘公干以失敬罹罪%
\footnote{%
    刘公干:
        刘桢,字公干,
        著名诗人,建安七子之一。
        曾随侍曹操的儿子曹丕(后即位,为魏文帝)。
        在一次宴会上,
        曹丕让夫人甄氏出来拜客,
        座上客人多拜伏在地,
        独独刘桢平视,
        这就是失敬。
        后来曹操知道了,
        把他逮捕下狱,
        判罚做苦工。
        按:
        刘桢获罪一事,
        发生在曹操当权时期,
        这里说成曹丕即帝位后,
        不确。
    罹(lí):遭受。
}%
。
文帝问曰:「
    卿何以不谨于文宪%
    \footnote{%
        文宪:法纪。
    }%
    ?
」
桢答曰:「
    臣诚庸短,
    亦由陛下纲目不疏。
」

%% ----------------------------------------------------------------------------
\switchcolumn

% %% Jy

% %% 妖
% 刘公干因为失敬获罪。
% 魏文帝问他:“
%  你为什么不遵守法纪呢?”
% 刘公干回答:“
%  罪臣才智平庸,目光短浅,
%  可陛下您的规章制度也不严密啊。”
